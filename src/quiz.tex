\chapter{Quiz}
\label{ch:quiz}

\begin{esercizio}[Gare distrettuali 2018]
    \label{ex:distrettuali_2018_1}
    Una gara di matematica consta di 90 domande a risposta multipla.
    Camilla ha risposto a tutte le domande: quale dei seguenti non può essere il punteggio totalizzato da Camilla,
    sapendo che una risposta corretta vale 5 punti e una risposta sbagliata vale $-1$ punto?

    (A) $-78$ \quad (B) 116 \quad (C) 204 \quad (D) 318 \quad (E) 402
\end{esercizio}

\begin{esercizio}[Gare distrettuali 2018]
    \label{ex:distrettuali_2018_15}
    \begin{enumerate}
        \item Trovare tutti gli interi positivi $n$ di due cifre che godano della seguente proprietà: entrambi
        gli interi che si ottengono cancellando una delle due cifre della rappresentazione decimale
        di $n$ sono divisori (interi positivi) di $n$.
        \item Sia $n > 10$ un intero che si scrive con $k$ cifre decimali, tutte diverse da zero.
        Supponiamo che ciascuno degli interi ottenuti cancellando una delle $k$ cifre della rappresentazione decimale
        di $n$ sia un divisore (intero positivo) di $n$.
        Mostrare che necessariamente $k = 2$. \\
        Esempio.
        Per $n = 123$ si ha $k = 3$, e gli interi ottenuti cancellando cifre di $n$ sono 23, 13 e 12.
    \end{enumerate}
\end{esercizio}

\begin{esercizio}[Gare distrettuali 2019]
    \label{ex:distrettuali_2019}
    Jacopo ha a disposizione 6 colori (tra cui il bianco) per colorare tutti i numeri interi.
    Vuole rispettare però queste condizioni: $n$ e $n + 5$ devono avere lo stesso colore per ogni $n$ intero e
    inoltre se $ab$ è bianco, allora almeno uno tra $a$ e $b$ deve essere bianco.
    In quanti modi Jacopo può colorare gli interi?

    (A) 156 \quad (B) 656 \quad (C) 3181 \quad (D) 3906 \quad (E) 3936
\end{esercizio}

\begin{esercizio}[Gare distrettuali 2021]
    \label{ex:distrettuali_2021}
    Quanti sono i numeri di 6 cifre divisibili per 33 che siano palindromi, cioè che rimangano uguali
    se letti da destra verso sinistra?

    (A) 30 \quad (B) 33 \quad (C) 300 \quad (D) 333 \quad (E) Nessuna delle precedenti.
\end{esercizio}

\begin{esercizio}[Proposto da Francesco P.]
    \label{ex:francesco_1}
    Quanti sono i numeri nella forma $\overline{abcabc}$ divisibili per 2023?
\end{esercizio}

\begin{esercizio}[Proposto da Francesco P.]
    \label{ex:francesco_2}
    Quanti sono i numeri nella forma $\overline{abcabc}$ che sono quadrati perfetti?
\end{esercizio}