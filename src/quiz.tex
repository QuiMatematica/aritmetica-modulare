\chapter{Esercizi e quiz}
\label{ch:quiz}

\section{Dall'eserciziario dello stage senior}
\label{sec:quiz_stage_senior}

\begin{esercizio}
    \label{ex:stage_senior_41}
    Determinare per quali valori interi di $n$ le seguenti espressioni sono intere.

    \begin{equation*}
        \dfrac{n + 3}{n + 1} \quad \dfrac{3n + 10}{n + 2} \quad \dfrac{n + 7}{2n + 1} \quad \dfrac{3a + 1}{2a + 3} \quad
        \dfrac{15 - 3n}{2n^2 + 1}
    \end{equation*}
\end{esercizio}

\section{Quiz dai giochi di Archimede}
\label{sec:quiz_giochi_archimede}

\begin{esercizio}[Giochi di Archimede 2000 - Triennio]
    \label{ex:archimede_2000_triennio_1}
    Un podista e un ciclista partono insieme dalla città $A$ diretti alla città $B$ distante da $A$ 13 km, con
    l’accordo di fare la spola fra $A$ e $B$ senza fermarsi mai.
    Sapendo che ogni ora il podista percorre 9 km mentre il ciclista ne percorre 25, quale distanza separerà i due
    sportivi dopo tre ore dall’inizio della competizione?

    (A) 1 km \quad (B) 2 km \quad (C) 3 km \quad (D) 4 km \quad (E) 5 km.
\end{esercizio}

\section{Quiz dalle gare distrettuali}
\label{sec:quiz_gare_distrettuali}

\begin{esercizio}[Gare distrettuali 2017]
    \label{ex:distrettuali_2017_13}
    Il ricco Creso compra 88 vasi identici.
    Il prezzo di ognuno di essi, espresso in dracme, è un numero intero (lo stesso per tutti gli 88 vasi).
    Sappiamo che Creso paga un totale di $a1211b$ dracme, dove $a$, $b$ sono cifre da determinare
    (e che possono essere distinte o meno).
    Quante dracme costa un singolo vaso?
\end{esercizio}

\begin{esercizio}[Gare distrettuali 2018]
    \label{ex:distrettuali_2018_1}
    Una gara di matematica consta di 90 domande a risposta multipla.
    Camilla ha risposto a tutte le domande: quale dei seguenti non può essere il punteggio totalizzato da Camilla,
    sapendo che una risposta corretta vale 5 punti e una risposta sbagliata vale $-1$ punto?

    (A) $-78$ \quad (B) 116 \quad (C) 204 \quad (D) 318 \quad (E) 402
\end{esercizio}

\begin{esercizio}[Gare distrettuali 2018]
    \label{ex:distrettuali_2018_15}
    \begin{enumerate}
        \item Trovare tutti gli interi positivi $n$ di due cifre che godano della seguente proprietà: entrambi
        gli interi che si ottengono cancellando una delle due cifre della rappresentazione decimale
        di $n$ sono divisori (interi positivi) di $n$.
        \item Sia $n > 10$ un intero che si scrive con $k$ cifre decimali, tutte diverse da zero.
        Supponiamo che ciascuno degli interi ottenuti cancellando una delle $k$ cifre della rappresentazione decimale
        di $n$ sia un divisore (intero positivo) di $n$.
        Mostrare che necessariamente $k = 2$. \\
        Esempio.
        Per $n = 123$ si ha $k = 3$, e gli interi ottenuti cancellando cifre di $n$ sono 23, 13 e 12.
    \end{enumerate}
\end{esercizio}

\begin{esercizio}[Gare distrettuali 2019]
    \label{ex:distrettuali_2019}
    Jacopo ha a disposizione 6 colori (tra cui il bianco) per colorare tutti i numeri interi.
    Vuole rispettare però queste condizioni: $n$ e $n + 5$ devono avere lo stesso colore per ogni $n$ intero e
    inoltre se $ab$ è bianco, allora almeno uno tra $a$ e $b$ deve essere bianco.
    In quanti modi Jacopo può colorare gli interi?

    (A) 156 \quad (B) 656 \quad (C) 3181 \quad (D) 3906 \quad (E) 3936
\end{esercizio}

\begin{esercizio}[Gare distrettuali 2021]
    \label{ex:distrettuali_2021}
    Quanti sono i numeri di 6 cifre divisibili per 33 che siano palindromi, cioè che rimangano uguali
    se letti da destra verso sinistra?

    (A) 30 \quad (B) 33 \quad (C) 300 \quad (D) 333 \quad (E) Nessuna delle precedenti.
\end{esercizio}

\begin{esercizio}[Gare distrettuali 2023, es. 9]
    \label{ex:distrettuali_2023_9}
    La successione $a_n$ è costruita nel modo seguente:
    $a_1$, $a_2$ sono interi compresi fra 1 e 9 (estremi inclusi);
    per $n \ge 3$, se la somma fra $a_{n-1}$ e $a_{n-2}$ consta di una sola cifra, allora tale somma è il valore di $a_n$;
    se invece $a_{n-1} + a_{n-2}$ ha più di una cifra, la somma delle sue cifre sarà il valore di $a_n$
    (ad esempio, se $a_4 = 7$ e $a_5 = 8$, allora $a_6 = 6$ in quanto $7 + 8 = 15$ e $1 + 5 = 6$).
    Quante sono le scelte possibili della coppia $(a_1, a_2)$ tali che si abbia $a_{2023} = 9$?

    (A) 1 \quad (B) 3 \quad (C) 9 \quad (D) 27 \quad (E) 81
\end{esercizio}

\section{Altri quiz}
\label{sec:quiz_altri}

\begin{esercizio}[Proposto da Francesco P]
    \label{ex:francesco_1}
    Quanti sono i numeri nella forma $\overline{abcabc}$ divisibili per 2023?
\end{esercizio}

\begin{esercizio}[Proposto da Francesco P]
    \label{ex:francesco_2}
    Quanti sono i numeri nella forma $\overline{abcabc}$ che sono quadrati perfetti?
\end{esercizio}

\begin{esercizio}[Proposto da Claudio S]
    \label{ex:claudio_1}
    Si consideri l'esercizio~\ref{ex:archimede_2000_triennio_1}.

    Scrivere un algoritmo che risolva il problema considerando una distanza tra le due città pari a $d$, una velocità
    del primo sportivo pari a $v_1$, una velocità del secondo sportivo pari a $v_2$ e un tempo pari a $t$.

    Dopo aver scritto l'algoritmo rifletti: l'algoritmo è adatto a qualunque valore dei parametri?
    O solo a valori interi?
    Se dovessi renderlo più generale e considerare valori reali, come dovresti modificare l'algoritmo?
\end{esercizio}