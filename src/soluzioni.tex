\chapter{Soluzioni}
\label{ch:soluzioni}


\begin{soluzione}{ex:stage_senior_41}

    Affinché $\dfrac{n + 3}{n + 1}$ sia intero è necessario che $n+1$ divida $n+3$:
    \begin{gather*}
        [n + 3]_{n + 1} = [0]_{n+1} \\
        [n + 1 + 2]_{n + 1} = [0]_{n+1} \\
        [n + 1]_{n+1} + [2]_{n + 1} = [0]_{n+1} \\
        [2]_{n + 1} = [0]_{n+1} \\
    \end{gather*}

    Quindi $n+1$ deve dividere 2.
    I divisori (positivi e negativi) di 2 sono $-2$, $-1$, 1 e 2:
    \begin{align*}
        n + 1 = -2 &\allora n = -3 \\
        n + 1 = -1 &\allora n = -2 \\
        n + 1 = 1 &\allora n = 0 \\
        n + 1 = 2 &\allora n = 1
    \end{align*}

    Affinché $\dfrac{3n + 10}{n + 2}$ sia intero è necessario che $n+2$ divida $3n+10$:
    \begin{gather*}
        [3n+10]_{n+2} = [0]_{n+2} \\
        [3n + 6 + 4]_{n+2} = [0]_{n+2} \\
        [3n + 6]_{n+2} + [4]_{n+2} = [0]_{n+2} \\
        [4]_{n+2} = [0]_{n+2}
    \end{gather*}

    Quindi $n+2$ deve dividere 4.
    I divisori (positivi e negativi) di 4 sono $-4$, $-2$, $-1$, 1, 2 e 4:
    \begin{align*}
        n+2 = -4 &\allora n = -6 \\
        n+2 = -2 &\allora n = -4 \\
        n+2 = -1 &\allora n = -3 \\
        n+2 = 1 &\allora n = -1 \\
        n+2 = 2 &\allora n = 0 \\
        n+2 = 4 &\allora n = 2
    \end{align*}

    Affinché $\dfrac{n + 7}{2n + 1}$ sia intero è necessario che $2n+1$ divida $n+7$.
    Però in questo caso non riusciamo ad isolare a numeratore un multiplo del denominatore, a causa di quel fattore 2
    che moltiplica $n$ a denominatore.

    Osserviamo che il denominatore è un numero dispari quindi se moltiplichiamo il numeratore per 2 non rischiamo
    di aggiungere fattori che sono presenti nel denominatore.
    Quindi $2n+1$ divide $n+7$ se e solo se $2n+1$ divide $2(n+7)$:
    \begin{gather*}
        [2n +14]_{2n+1} = [0]_{2n + 1} \\
        [13]_{2n + 1} = [0]_{2n + 1}
    \end{gather*}

    I divisori di 13 sono $-13$, $-1$, 1 e 13:
    \begin{align*}
        2n + 1 = -13 &\allora n = -7 \\
        2n + 1 = -1 &\allora n = -1 \\
        2n + 1 = 1 &\allora n = 0 \\
        2n + 1 = 13 &\allora n = 6
    \end{align*}

    Affinché $\dfrac{3a + 1}{2a + 3}$ sia intero è necessario che $2a+3$ divida $3a+1$.
    Il denominatore è dispari, quindi posso moltiplicare il numeratore per 2:
    \begin{gather*}
        [6a + 2]_{2a + 3} = [0]_{2a + 3} \\
        [6a + 9 - 7]_{2a + 3} = [0]_{2a+3} \\
        [-7]_{2a + 3} = [0]_{2a+3}
    \end{gather*}

    Quindi $2a + 3$ deve dividere 7:
    \begin{align*}
        2a + 3 = -7 &\allora a = -5 \\
        2a + 3 = -1 &\allora a = -2 \\
        2a + 3 = 1 &\allora a = -1 \\
        2a + 3 = 7 &\allora a = 2
    \end{align*}

    Per l'ultimo caso dobbiamo ricorrere a qualche altro stratagemma, perché il denominatore scritto in quel modo
    non mi permette di operare semplificazioni sul numeratore.

    Possiamo allora tentare di restringere l'analisi tenendo conto del fatto che il divisore deve essere (in valore
    assoluto) minore o uguale al denominatore (in valore assoluto).
    Nella frazione considerata, inoltre, abbiamo il vantaggio che il denominatore è sicuramente positivo, quindi
    possiamo semplificare la condizione in:
    \begin{gather*}
        2n^2 + 1 \le 15 - 3n \\
        2n^2 + 3n - 14 \le 0 \\
        -\dfrac{7}{2} \le n \le 2
    \end{gather*}

    Quindi gli unici casi da analizzare sono i seguenti:
    \begin{align*}
        n = -3 &\allora \dfrac{15 - 3n}{2n^2 + 1} = \dfrac{24}{19} &\text{non accettabile} \\
        n = -2 &\allora \dfrac{15 - 3n}{2n^2 + 1} = \dfrac{21}{9} &\text{non accettabile} \\
        n = -1 &\allora \dfrac{15 - 3n}{2n^2 + 1} = \dfrac{18}{3} &\text{accettabile} \\
        n = 0 &\allora \dfrac{15 - 3n}{2n^2 + 1} = \dfrac{15}{1} &\text{accettabile} \\
        n = 1 &\allora \dfrac{15 - 3n}{2n^2 + 1} = \dfrac{12}{3} &\text{accettabile} \\
        n = 2 &\allora \dfrac{15 - 3n}{2n^2 + 1} = \dfrac{9}{9} &\text{accettabile}
    \end{align*}
\end{soluzione}

\begin{soluzione}{ex:distrettuali_2017_13}
    Dobbiamo determinare le cifre $a$ e $b$ tali che:
    \begin{equation*}
        a1211b \modulo 88 = 0
    \end{equation*}

    Un numero è divisibile per 88 se è divisibile per 8 e per 11.
    Quindi il problema è equivalente a determinare le cifre tali che:
    \begin{equation*}
        \begin{cases}
            a1211b \modulo 8 = 0 \\
            a1211b \modulo 11 = 0
        \end{cases}
    \end{equation*}

    Passiamo alle classi di congruenza e, sfruttando i criteri di congruenza, abbiamo:
    \begin{gather*}
        [a1211b]_8 = [0]_8 \\
        [11b]_8 = [0]_8 \\
        [100]_8 + [10]_8 + [b]_8 = [0]_8 \\
        [4]_8 + [2]_8 + [b]_8 = [0]_8 \\
        [6]_8 + [b]_8 = [0]_8 \\
        [b]_8 = -[6]_8 \\
        [b]_8 = [2]_8
    \end{gather*}

    L'unica cifra singola che appartiene alla classe di congruenza $[2]_8$ è il 2, quindi $b = 2$.

    Verifichiamo ora che il numero $a12112$ sia divisibile per 11.

    Balza all'occhio che il numero $112112$ è divisibile per 11, e visto che il problema sembra prevedere un'unica
    soluzione, potremo dare questa risposta.
    Ma assicuriamoci che non esistano altre soluzioni.
    Passiamo alle classi di congruenza modulo 11 e, nuovamente sfruttando i criteri di congruenza, abbiamo:
    \begin{gather*}
        [a12112]_{11} = [0]_{11} \\
        [-a + 1 - 2 + 1 - 1 + 2]_{11} = [0]_{11} \\
        [-a + 1]_{11} = [0]_{11} \\
        -[a]_{11} + [1]_{11} = [0]_{11} \\
        [a]_{11} = [1]_{11}
    \end{gather*}

    L'unica cifra singola che appartiene alla classe di congruenza $[1]_{11}$ è l'1, quindi $a = 1$.

    Non ci resta che fare la divisione:
    \begin{equation*}
        112112 : 88 = 1274
    \end{equation*}

    \textbf{Un singolo vaso costa 1274 dracme.}
\end{soluzione}

\begin{soluzione}{ex:distrettuali_2018_1}
    Chiamiamo $n$ il numero di risposte corrette date da Camilla.
    Quindi ha sbagliato $90 - n$ risposte.

    Per le risposte corrette ottiene $5n$ punti.

    Per le risposte sbagliate ottiene $-1(90-n)$ punti.

    In totale ottiene:
    \[
        5n - 1(90 - n) = 5n - 90 + n = 6n - 90 = 6(n - 15)
    \]

    Quindi il punteggio di Camilla deve essere divisibile per 6.
    L'unico punteggio tra quelli proposti non divisibile per 6 è il 116.
\end{soluzione}

\begin{soluzione}{ex:distrettuali_2018_15}

    \textbf{Parte 1.}

    L'esercizio ci chiede di trovare tutti i numeri interi positivi di due cifre $\overline{ab}$ che sono divisibili
    per $a$ e per $b$:
    \begin{gather*}
        a \divisore \overline{ab} \\
        b \divisore \overline{ab}
    \end{gather*}

    Il divisore non può essere nullo, quindi $a$ e $b$ non sono nulli.

    Passando alle classi di congruenza, per quanto riguarda il divisore $a$ abbiamo che:
    \begin{gather*}
        a \divisore \overline{ab} \\
        [\overline{ab}]_a = [0]_a \\
        [10]_a[a]_a + [b]_a = [0]_a \\
        [10]_a[0]_a + [b]_a = [0]_a \\
        [b]_a = [0]_a \\
        a \divisore b \\
        \exists k : b = ka
    \end{gather*}

    Invece per quanto riguarda il divisore $b$ abbiamo che:
    \begin{gather*}
        b \divisore \overline{ab} \\
        [\overline{ab}]_b = [0]_b \\
        [10]_b[a]_b + [b]_b = [0]_b \\
        [10]_b[a]_b + [0]_b = [0]_b \\
        [10]_b[a]_b = [0]_b \\
        [10a]_b = [0]_b \\
        b \divisore 10a \\
        ka \divisore 10a \\
        k \divisore 10
    \end{gather*}

    Riassumendo abbiamo ottenuto due condizioni che dobbiamo rispettare:
    \begin{gather*}
        b = ka \\
        k \divisore 10
    \end{gather*}

    I valori possibili di $k$ sono i divisori di 10 che fanno in modo che $b$ sia una cifra.
    Essi sono: 1, 2 e 5.

    Enumeriamo quindi i casi possibili:
    \begin{itemize}
        \item $k = 1 \,\Longrightarrow\, b = a$: ottengo i numeri 11, 22, 33, 44, 55, 66, 77, 88, 99;
        \item $k = 2 \,\Longrightarrow\, b = 2a$: ottengo i numeri 12, 24, 36, 48;
        \item $k = 5 \,\Longrightarrow\, b = 5a$: ottengo il numero 15.
    \end{itemize}

    \bigskip
    \textbf{Gli interi positivi cercati sono: 11, 12, 15, 22, 24, 33, 36, 44, 48, 55, 66, 77, 88, 99.}

    \bigskip
    \textbf{Parte 2.}
    Ora abbiamo a che fare con numeri di due o più cifre.
    Non possiamo più usare la notazione posizionale, ma possiamo comunque scomporre il numero $n$ in $10a + b$,
    dove $a$ è un qualunque numero intero maggiore di 1, e $b$ è una singola cifra.

    Il numero $n = 10a + b$ deve essere divisibile per $a$, quindi:
    \begin{gather*}
        a \divisore 10a + b \\
        [10a + b]_a = [0]_a \\
        [10]_a[a]_a + [b]_a = [0]_a \\
        [10]_a[0]_a + [b]_a = [0]_a \\
        [b]_a = [0]_a \\
        a \divisore b
    \end{gather*}

    Affinché $a$ sia un divisore di $b$ è necessario che $a \le b$.
    Visto che $b$ è una singola cifra, ovvero $b \le 9$, deduciamo che anche $a \le 9$.
    Quindi il numero numero $n = 10a + b$ è composto al più di 2 cifre.
\end{soluzione}

\begin{soluzione}{ex:distrettuali_2019}
    La condizione
    \begin{quotation}
        $n$ e $n + 5$ devono avere lo stesso colore per ogni $n$ intero
    \end{quotation}
    ci permette di operare considerando solo le classi di congruenza modulo 5.
    Infatti $n$ e $n + 5$ si trovano nella stessa classe:
    \begin{align*}
        [n + 5]_5 &= [n]_5 + [5]_5 = && \text{per somma di classi} \\
        &= [n]_5 + [0]_5 = && \text{perché $5 \modulo 5 = 0$} \\
        &= [n]_5 && \text{perché $[0]_5$ è l'elemento neutro della somma}
    \end{align*}

    Dal momento che in questo esercizio avremo a che fare solo con classi modulo 5 non verrà più specificato il modulo
    nelle notazioni.

    Gli elementi di ciascuna classe di congruenza avranno tutti lo stesso colore.
    Dobbiamo quindi determinare in quanti modi possiamo colorare le 5 classi.
    Nella colorazione dobbiamo rispettare anche la seconda condizione:
    \begin{quotation}
        se $ab$ è bianco, allora almeno uno tra $a$ e $b$ deve essere bianco.
    \end{quotation}

    Le classi di congruenza si comportano bene con il prodotto tra numeri, quindi posso modificare la condizione in:
    \begin{quotation}
        se $[ab]$ è bianca, allora almeno una tra $[a]$ e $[b]$ deve essere bianca.
    \end{quotation}

    Abbiamo quindi bisogno di capire come si comportano le moltiplicazioni con le classi di congruenza modulo 5.
    Costruiamo quindi la tavola pitagorica (o \emph{tavola di Cayley}) del prodotto tra le classi di congruenza:

    \begin{table}[H]
        \label{tab:distrettuali_2019}
        \centering
        \begin{tabular}{c|ccccc}
            $\cdot$ & $[0]$ & $[1]$ & $[2]$ & $[3]$ & $[4]$ \\
            \midrule
            $[0]$ & $[0]$ & $[0]$ & $[0]$ & $[0]$ & $[0]$ \\
            $[1]$ & $[0]$ & $[1]$ & $[2]$ & $[3]$ & $[4]$ \\
            $[2]$ & $[0]$ & $[2]$ & $[4]$ & $[1]$ & $[3]$ \\
            $[3]$ & $[0]$ & $[3]$ & $[1]$ & $[4]$ & $[2]$ \\
            $[4]$ & $[0]$ & $[4]$ & $[3]$ & $[2]$ & $[1]$
        \end{tabular}
    \end{table}

    Questa tabella ci permette di capire con quali prodotti otteniamo le classi di congruenza:
    \begin{itemize}
        \item $[0] = [0][a] \,\,\forall [a]$;
        \item $[1] = [1][1] = [2][3] = [4][4]$;
        \item $[2] = [1][2] = [3][4]$;
        \item $[3] = [1][3] = [2][4]$;
        \item $[4] = [1][4] = [2][2] = [3][3]$.
    \end{itemize}

    Lo schema appena fatto ci dice, per esempio, che se noi moltiplichiamo due elementi qualunque della classe $[4]$
    otteniamo un elemento della classe $[1]$ (per esempio: $9 \cdot 14 = 126$).
    Se invece moltiplichiamo un qualunque elemento della classe $[2]$ con un qualunque elemento delle classe $[4]$
    otteniamo un elemento della classe $[3]$ (per esempio: $7 \cdot 9 = 63$)

    Ora possiamo capire come dobbiamo comportarci se coloriamo di bianco una delle classi di congruenza:
    \begin{itemize}
        \item se coloriamo di bianco la classe $[0]$ possiamo colorare le altre classi di qualunque altro colore,
        e il vincolo è rispettato;
        \item se coloriamo di bianco la classe $[1]$ dobbiamo colorare di bianco anche la classe $[4]$ (visto che otteniamo un
        elemento di $[1]$ moltiplicando tra loro due qualunque elementi di $[4]$) e, di conseguenza, dobbiamo colorare
        di bianco anche le classi $[2]$ e $[3]$ (visto che otteniamo un elemento di $[4]$ moltiplicando due elementi di
        $[2]$ o due elementi di $[3]$); quindi se coloro di bianco $[1]$ devo colorare anche $[2]$, $[3]$ e $[4]$;
        \item se coloriamo di bianco la classe $[2]$ dobbiamo colorare di bianco anche la classe $[3]$ oppure la classe
        $[4]$; nel primo caso non ho bisogno di colorare altro di bianco;
        nel secondo caso devo colorare di bianco anche la classe $[3]$;
        \item se coloriamo di bianco la classe $[3]$ dobbiamo colorare di bianco anche la classe $[2]$ oppure la classe
        $[4]$; nel primo caso non ho bisogno di colorare altro di bianco;
        nel secondo caso devo colorare di bianco anche la classe $[2]$;
        \item se coloriamo di bianco la classe $[4]$ dobbiamo colorare di bianco anche la classe $[2]$ e $[3]$.
    \end{itemize}

    Riassumendo, abbiamo le seguenti situazioni:
    \begin{itemize}
        \item coloriamo di bianco le classi $[1]$, $[2]$, $[3]$ e $[4]$:
        possiamo colorare la classe $[0]$ di qualunque colore;
        abbiamo quindi 6 possibili scelte legate al colore che daremo alla classe $[0]$;
        \item coloriamo di bianco le classi $[2]$, $[3]$ e $[4]$ e la classe $[1]$ di un colore che non sia bianco;
        anche in questo caso possiamo colorare la classe $[0]$ di qualunque colore (compreso il bianco);
        abbiamo quindi 6 possibili scelte per il colore di $[0]$ e 5 possibile scelte (escludiamo infatti il bianco)
        per la classe $[1]$;
        \item coloriamo di bianco le classi $[2]$ e $[3]$ e le classi $[1]$ e $[4]$ di colori che non siano il bianco;
        anche in questo caso possiamo colorare la classe $[0]$ di qualunque colore (compreso il bianco);
        abbiamo quindi 6 possibili scelte per il colore di $[0]$, 5 possibili scelte per $[1]$ e 5 possibili scelte
        per $[4]$;
        \item coloriamo le classi $[1]$, $[2]$, $[3]$ e $[4]$ di una qualunque combinazione di colori che non includa
        il bianco;
        siamo ancora liberi di colorare $[0]$ con qualunque colore, compreso il bianco;
        abbiamo quindi 6 possibili scelte per il colore di $[0]$, 5 per $[1]$, 5 per $[2]$, 5 per $[3]$ e 5 per $[4]$.
    \end{itemize}

    La cosa importante è avere coscienza che le quattro casistiche viste sopra sono \emph{disgiunte}, ovvero non c'è
    nessun caso che compare più di una volta.
    Se i casi non fossero disgiunti, starei contando una possibilità più volte.

    Visto che i casi sono disgiunti, possiamo sommare tutte le possibili scelte:
    \begin{multline*}
        6 + 6 \cdot 5 + 6 \cdot 5 \cdot 5 + 6 \cdot 5 \cdot 5 \cdot 5 \cdot 5 = \\
        = 6(1 + 5 + 5^2 + 5^4) = 6(1 + 5 + 25 + 625) = 6 \cdot 656 = \\
        = 3936
    \end{multline*}

    \bigskip
    \textbf{Jacopo può colorare gli interi in 3936 modi diversi.}
\end{soluzione}

\begin{soluzione}{ex:distrettuali_2021}
    I numeri palindromi di 6 cifre hanno la seguente struttura:

    \begin{equation*}
        \overline{abccba}
    \end{equation*}

    Dobbiamo trovare quanti di questi numeri sono divisibili per 33, ovvero sono divisibili per 3 e per 11.

    Innanzitutto osserviamo che tutti i numeri palindromi di 6 cifre sono divisibili per 11, infatti, per il
    criterio di congruenza modulo 11:

    \begin{equation*}
        [\overline{abccba}]_{11} = [-a + b - c + a - b + c]_{11} = [0]_{11}
    \end{equation*}

    Dobbiamo quindi determinare quanti numeri palindromi di 6 cifre sono divisibili per 3.
    Per il criterio di congruenza modulo 3 risulta:
    \begin{align*}
        [\overline{abccba}]_3 &= [a + b + c + c + b + a]_3 = \\
        &= [2a + 2b + 2c]_3 = \\
        &= [2(a+b+c)]_3 = \\
        &= [2]_3[a + b + c]_3 \\
        &= [2]_3[\overline{abc}]_3
    \end{align*}

    Quindi se vogliamo che $\overline{abccba}$ sia divisibile per 3, visto che 2 non lo è, è necessario che $a + b + c$
    sia divisibile per 3.

    A questo punto ti propongo due soluzioni.

    \bigskip
    \textbf{Soluzione velocissima}

    Dobbiamo determinare quanti sono i numeri di tre cifre che sono divisibili per 3.

    I numeri di 3 cifre vanno da 100 a 999 e in totale solo 900.
    Solo uno ogni tre di questi numeri è divisibile per 100, quindi quelli divisibili per 3 sono:

    \begin{equation*}
        900 : 3 = 300
    \end{equation*}

    \bigskip
    \textbf{Soluzione più lunga}

    Continuando con il criterio di congruenza modulo 3 possiamo scrivere:

    \begin{equation*}
        [a + b + c]_3 = [a]_3 + [b]_3 + [c]_3
    \end{equation*}

    Affinché tale numero sia divisibile per 3 dobbiamo scegliere dei valori di $a$, $b$ e $c$ tali che:

    \begin{equation*}
        [a]_3 + [b]_3 + [c]_3 = [0]_3
    \end{equation*}

    Dobbiamo ora andare a contare quante solo le combinazioni di $a$, $b$ e $c$ che soddisfano tale vincolo.
    Anzi: lo
    faremo in due passaggi: prima contiamo le combinazioni delle classi di congruenza $[a]_3$, $[b]_3$ e $[c]_3$ e poi
    da queste passiamo alle cifre.

    Nota che scelta la classe di congruenza per $[a]_3$ e per $[b]_3$, resta automaticamente determinata la classe di
    congruenza per $[c]_3$.
    Per esempio, se $[a]_3 = [1]_3$ e $[b]_3 = [0]$, allora $[c]_3 =  [2]_3$, infatti:

    \begin{equation*}
        [1]_3 + [0]_3 + [2]_3 = [3]_3 = [0]_3
    \end{equation*}

    Allora le possibili combinazioni (considerando le classi di congruenza) sono date dalle 3 possibili scelte di $[a]_3$
    e dalle 3 possibili scelte di $[b]_3$:

    \begin{table}[H]
        \label{tab:distrettuali_2019_1}
        \centering
        \begin{tabular}{ccc}
            \toprule
            $[a]_3$ & $[b]_3$ & $[c]_3$ \\
            \midrule
            $[0]_3$ & $[0]_3$ & $[0]_3$ \\
            $[0]_3$ & $[1]_3$ & $[1]_3$ \\
            $[0]_3$ & $[2]_3$ & $[2]_3$ \\
            $[1]_3$ & $[0]_3$ & $[2]_3$ \\
            $[1]_3$ & $[1]_3$ & $[1]_3$ \\
            $[1]_3$ & $[2]_3$ & $[0]_3$ \\
            $[2]_3$ & $[0]_3$ & $[1]_3$ \\
            $[2]_3$ & $[1]_3$ & $[0]_3$ \\
            $[2]_3$ & $[2]_3$ & $[2]_3$ \\
            \bottomrule
        \end{tabular}
    \end{table}

    Ora sostituiamo le classi di congruenza con la quantità di scelte che abbiamo per ciascuna di esse.
    Ovviamente non conteremo tutti gli infiniti numeri interi che appartengono alle classi di congruenza, ma solo le
    cifre singole.
    Dobbiamo quindi tenere conto che:
    \begin{itemize}
        \item $[0]_3 = \{0, 3, 6, 8\}$ contiene 4 elementi, ma se usata per la prima cifra $[a]_3$ non possiamo
        considerare la cifra 0 altrimenti avremo un numero di sole 5 cifre;
        \item $[1]_3 = \{1, 4, 7\}$ contiene 3 elementi;
        \item $[2]_3 = \{2, 5, 8\}$ contiene 3 elementi.
    \end{itemize}

    Inoltre, per il principio di moltiplicazione, le possibili scelte per ogni riga saranno date dal prodotto delle
    scelte delle singole cifre.
    Quindi:

    \begin{table}[H]
        \label{tab:distrettuali_2019_2}
        \centering
        \begin{tabular}{ccccccc}
            \toprule
            $[a]_3$ & $[b]_3$ & $[c]_3$ & scelte di $[a]_3$ & scelte di $[b]_3$ & scelte di $[c]_3$ & totali \\
            \midrule
            $[0]_3$ & $[0]_3$ & $[0]_3$ & 3 & 4 & 4 & 48 \\
            $[0]_3$ & $[1]_3$ & $[1]_3$ & 3 & 3 & 3 & 27 \\
            $[0]_3$ & $[2]_3$ & $[2]_3$ & 3 & 3 & 3 & 27 \\
            $[1]_3$ & $[0]_3$ & $[2]_3$ & 3 & 4 & 3 & 36 \\
            $[1]_3$ & $[1]_3$ & $[1]_3$ & 3 & 3 & 3 & 27 \\
            $[1]_3$ & $[2]_3$ & $[0]_3$ & 3 & 3 & 4 & 36 \\
            $[2]_3$ & $[0]_3$ & $[1]_3$ & 3 & 4 & 3 & 36 \\
            $[2]_3$ & $[1]_3$ & $[0]_3$ & 3 & 3 & 4 & 36 \\
            $[2]_3$ & $[2]_3$ & $[2]_3$ & 3 & 3 & 3 & 27 \\
            \midrule
            & & & & & & 300 \\
            \bottomrule
        \end{tabular}
    \end{table}

    \bigskip
    \textbf{I numeri palindromi di 6 cifre divisibili per 33 sono 300.}
\end{soluzione}

\begin{soluzione}{ex:distrettuali_2023_9}
    Innanzitutto osserviamo che il modo di costruire la successione utilizza le classi di congruenza modulo 9, dove
    però viene scelto il numero 9 come rappresentante della classe $[0]_9$.

    Infatti se abbiamo due numeri $a$ e $b$ la cui somma $a+b$ è minore di 10 (quindi è composta da una sola cifra),
    $a+b$ può essere considerata come la somma delle classi di congruenza dei due addendi $a$ e $b$.
    Quindi:
    \begin{equation*}
        a + b < 10 \allora [a]_9 + [b]_9 = [a+b]_9
    \end{equation*}

    Se invece la somma è maggiore o uguale a 10 (quindi composta di più di una cifra), il fare la somma delle cifre
    significa utilizzare il criterio di congruenza modulo 9.
    E quindi:
    \begin{equation*}
        a + b \ge 10 \allora [a]_9 + [b]_9 = [a+b]_9
    \end{equation*}

    Insomma: invece che lavorare con i numeri lavoriamo con le classi di congruenza modulo 9 (che non specificherò più).

    Ora, partiamo dal fondo e, invece di scegliere i numeri di partenza $a_1$ e $a_2$, vediamo in quanti modi possiamo
    scegliere i numeri $a_{2022}$ e $a_{2021}$.
    E ricordiamoci che dobbiamo ottenere $a_{2023} = [9] = [0]$

    Iniziamo scegliendo $a_{2022} = [0]$.
    Quali possono essere i valori di $a_{2021}$?
    Può essere uno solo, perché solo se $a_{2021} = [0]$ abbiamo:
    \begin{equation*}
        a_{2023} = a_{2021} + a_{2022} = [0] + [0] = [0]
    \end{equation*}

    Ma se $a_{2022} = [0]$ e $a_{2021} = [0]$, allora $a_{2020}$ non può che essere $[0]$.
    E così via fino ad arrivare a $a_1 = [0]$ e $a_2 = [0]$.

    Ora scegliamo $a_{2022} = [1]$.
    Quali possono essere i valori di $a_{2021}$?
    Può essere solo $a_{2021} = [8]$, perché
    \begin{equation*}
        a_{2023} = a_{2021} + a_{2022} = [8] + [1] = [0]
    \end{equation*}

    Ma se $a_{2022} = [1]$ e $a_{2021} = [8]$, allora $a_{2020}$ non può che essere $[2]$.
    Se seguo tutta la catena a ritroso fino a $a_1$ ho sempre una sola scelta per assegnare un valore a ciascun elemento
    della successione.
    Non ho idea di quale sia (e non mi interessa saperlo) la coppia di partenza $(a_1, a_2)$, ma sicuramente c'è una sola
    coppia che porta ad avere $a_{2022} = [1]$.

    A questo punto diventa immediato capire che qualunque sia il numero che scegliamo per $a_{2022}$ c'è un'unica
    successione di numeri che porta a quel valore.
    Ed in particolare per ogni possibile valore di $a_{2022}$ c'è una sola coppia $(a_1, a_2)$ di partenza.

    Abbiamo 9 possibili valori di $a_{2022}$, quindi abbiamo 9 possibili coppie di partenza. \textbf{Risposta C.}

\end{soluzione}

\begin{soluzione}{ex:francesco_1}
    Il numero $\overline{abcabc}$ può essere scomposto in:
    \begin{equation*}
        \overline{abcabc} = \overline{abc} \cdot 1001 = \overline{abc} \cdot 7 \cdot 11 \cdot 13
    \end{equation*}

    Il divisore 2023 invece si scompone in:
    \begin{equation*}
        2023 = 7 \cdot 17^2
    \end{equation*}

    Se 2023 divide $\overline{abcabc}$ significa allora che:
    \begin{equation*}
        \exists k \in \Z: \overline{abc} \cdot 7 \cdot 11 \cdot 13 = k \cdot 7 \cdot 17^2
    \end{equation*}

    Possiamo semplificare per 7:
    \begin{equation*}
        \exists k \in \Z: \overline{abc} \cdot 11 \cdot 13 = k \cdot 17^2
    \end{equation*}

    Non abbiamo altri fattori in comune da semplificare, quindi è necessario che $\overline{abc}$ sia divisibile per
    $17^2$.

    Dobbiamo quindi determinare quanti numeri di tre cifre sono divisibili per $17^2 = 289$.

    289 è maggiore di 250, quindi i numeri cercati sono meno di 4:
    \begin{equation*}
        1000 : 289 < 1000 : 250 = 4
    \end{equation*}

    289 è minore di $333 \sim \frac{1000}{3}$, quindi ci sono almeno 3 multipli di 3 cifre:
    \begin{equation*}
        1000 : 289 > 1000 : \frac{1000}{3} = 3
    \end{equation*}

    \bigskip
    \textbf{I numeri nella forma $\overline{abcabc}$ divisibili per 2023 sono 3: 289289, 578578 e 867867.}
\end{soluzione}

\begin{soluzione}{ex:francesco_2}
    Il numero $\overline{abcabc}$ può essere scomposto in:

    \begin{equation*}
        \overline{abcabc} = \overline{abc} \cdot 1001 = \overline{abc} \cdot 7 \cdot 11 \cdot 13
    \end{equation*}

    Affinché $\overline{abcabc}$ sia un quadrato perfetto è quindi necessario che $\overline{abc}$ sia divisibile per
    7, 11 e 13, ovvero deve essere divisibile per 1001.
    Ma il più piccolo numero (diverso da 0) divisibile per 1001 è 1001, che è composto di 4 cifre, non di 3.

    \bigskip
    \textbf{Non esistono numeri nella forma $\overline{abcabc}$ che siano quadrati perfetti.}
\end{soluzione}
