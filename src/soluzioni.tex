\chapter{Soluzioni}
\label{ch:soluzioni}

\begin{soluzione}
    \mdfsubtitle{Soluzione esercizio~\ref{ex:distrettuali_2019}}
    La condizione
    \begin{quotation}
        $n$ e $n + 5$ devono avere lo stesso colore per ogni $n$ intero
    \end{quotation}
    ci permette di operare considerando solo le classi di congruenza modulo 5.
    Infatti $n$ e $n + 5$ si trovano nella stessa classe:
    \begin{align*}
        [n + 5]_5 &= [n]_5 + [5]_5 = && \text{per somma di classi} \\
        &= [n]_5 + [0]_5 = && \text{perché $5 \modulo 5 = 0$} \\
        &= [n]_5 && \text{perché $[0]_5$ è l'elemento neutro della somma}
    \end{align*}

    Dal momento che in questo esercizio avremo a che fare solo con classi modulo 5 non verrà più specificato il modulo
    nelle notazioni.

    Gli elementi di ciascuna classe di congruenza avranno tutti lo stesso colore.
    Dobbiamo quindi determinare in quanti modi possiamo colorare le 5 classi.
    Nella colorazione dobbiamo rispettare anche la seconda condizione:
    \begin{quotation}
        se $ab$ è bianco, allora almeno uno tra $a$ e $b$ deve essere bianco.
    \end{quotation}

    Le classi di congruenza si comportano bene con il prodotto tra numeri, quindi posso modificare la condizione in:
    \begin{quotation}
        se $[ab]$ è bianca, allora almeno una tra $[a]$ e $[b]$ deve essere bianca.
    \end{quotation}

    Abbiamo quindi bisogno di capire come si comportano le moltiplicazioni con le classi di congruenza modulo 5.
    Costruiamo quindi la tavola pitagorica (o \emph{tavola di Cayley}) del prodotto tra le classi di congruenza:

    \begin{table}[H]
        \label{tab:distrettuali_2019}
        \centering
        \begin{tabular}{c|ccccc}
            $\cdot$ & $[0]$ & $[1]$ & $[2]$ & $[3]$ & $[4]$ \\
            \midrule
            $[0]$ & $[0]$ & $[0]$ & $[0]$ & $[0]$ & $[0]$ \\
            $[1]$ & $[0]$ & $[1]$ & $[2]$ & $[3]$ & $[4]$ \\
            $[2]$ & $[0]$ & $[2]$ & $[4]$ & $[1]$ & $[3]$ \\
            $[3]$ & $[0]$ & $[3]$ & $[1]$ & $[4]$ & $[2]$ \\
            $[4]$ & $[0]$ & $[4]$ & $[3]$ & $[2]$ & $[1]$
        \end{tabular}
    \end{table}

    Questa tabella ci permette di capire con quali prodotti otteniamo le classi di congruenza:
    \begin{itemize}
        \item $[0] = [0][a] \,\,\forall [a]$;
        \item $[1] = [1][1] = [2][3] = [4][4]$;
        \item $[2] = [1][2] = [3][4]$;
        \item $[3] = [1][3] = [2][4]$;
        \item $[4] = [1][4] = [2][2] = [3][3]$.
    \end{itemize}

    Lo schema appena fatto ci dice, per esempio, che se noi moltiplichiamo due elementi qualunque della classe $[4]$
    otteniamo un elemento della classe $[1]$ (per esempio: $9 \cdot 14 = 126$).
    Se invece moltiplichiamo un qualunque elemento della classe $[2]$ con un qualunque elemento delle classe $[4]$
    otteniamo un elemento della classe $[3]$ (per esempio: $7 \cdot 9 = 63$)

    Ora possiamo capire come dobbiamo comportarci se coloriamo di bianco una delle classi di congruenza:
    \begin{itemize}
        \item se coloriamo di bianco la classe $[0]$ possiamo colorare le altre classi di qualunque altro colore,
        e il vincolo è rispettato;
        \item se coloriamo di bianco la classe $[1]$ dobbiamo colorare di bianco anche la classe $[4]$ (visto che otteniamo un
        elemento di $[1]$ moltiplicando tra loro due qualunque elementi di $[4]$) e, di conseguenza, dobbiamo colorare
        di bianco anche le classi $[2]$ e $[3]$ (visto che otteniamo un elemento di $[4]$ moltiplicando due elementi di
        $[2]$ o due elementi di $[3]$); quindi se coloro di bianco $[1]$ devo colorare anche $[2]$, $[3]$ e $[4]$;
        \item se coloriamo di bianco la classe $[2]$ dobbiamo colorare di bianco anche la classe $[3]$ oppure la classe
        $[4]$; nel primo caso non ho bisogno di colorare altro di bianco;
        nel secondo caso devo colorare di bianco anche la classe $[3]$;
        \item se coloriamo di bianco la classe $[3]$ dobbiamo colorare di bianco anche la classe $[2]$ oppure la classe
        $[4]$; nel primo caso non ho bisogno di colorare altro di bianco;
        nel secondo caso devo colorare di bianco anche la classe $[2]$;
        \item se coloriamo di bianco la classe $[4]$ dobbiamo colorare di bianco anche la classe $[2]$ e $[3]$.
    \end{itemize}

    Riassumendo, abbiamo le seguenti situazioni:
    \begin{itemize}
        \item coloriamo di bianco le classi $[1]$, $[2]$, $[3]$ e $[4]$:
        possiamo colorare la classe $[0]$ di qualunque colore;
        abbiamo quindi 6 possibili scelte legate al colore che daremo alla classe $[0]$;
        \item coloriamo di bianco le classi $[2]$, $[3]$ e $[4]$ e la classe $[1]$ di un colore che non sia bianco;
        anche in questo caso possiamo colorare la classe $[0]$ di qualunque colore (compreso il bianco);
        abbiamo quindi 6 possibili scelte per il colore di $[0]$ e 5 possibile scelte (escludiamo infatti il bianco)
        per la classe $[1]$;
        \item coloriamo di bianco le classi $[2]$ e $[3]$ e le classi $[1]$ e $[4]$ di colori che non siano il bianco;
        anche in questo caso possiamo colorare la classe $[0]$ di qualunque colore (compreso il bianco);
        abbiamo quindi 6 possibili scelte per il colore di $[0]$, 5 possibili scelte per $[1]$ e 5 possibili scelte
        per $[4]$;
        \item coloriamo le classi $[1]$, $[2]$, $[3]$ e $[4]$ di una qualunque combinazione di colori che non includa
        il bianco;
        siamo ancora liberi di colorare $[0]$ con qualunque colore, compreso il bianco;
        abbiamo quindi 6 possibili scelte per il colore di $[0]$, 5 per $[1]$, 5 per $[2]$, 5 per $[3]$ e 5 per $[4]$.
    \end{itemize}

    La cosa importante è avere coscienza che le quattro casistiche viste sopra sono \emph{disgiunte}, ovvero non c'è
    nessun caso che compare più di una volta.
    Se i casi non fossero disgiunti, starei contando una possibilità più volte.

    Visto che i casi sono disgiunti, possiamo sommare tutte le possibili scelte:
    \begin{multline*}
        6 + 6 \cdot 5 + 6 \cdot 5 \cdot 5 + 6 \cdot 5 \cdot 5 \cdot 5 \cdot 5 = \\
        = 6(1 + 5 + 5^2 + 5^4) = 6(1 + 5 + 25 + 625) = 6 \cdot 656 = \\
        = 3936
    \end{multline*}

    \bigskip
    \mdfsubtitle{Risposta}
    Jacopo può colorare gli interi in 3936 modi diversi.
\end{soluzione}

\begin{soluzione}
    \mdfsubtitle{Soluzione esercizio~\ref{ex:distrettuali_2018}: parte 1)}
    L'esercizio ci chiede di trovare tutti i numeri interi positivi di due cifre $\overline{ab}$ che sono divisibili
    per $a$ e per $b$:
    \begin{gather*}
        a \divisore \overline{ab} \\
        b \divisore \overline{ab}
    \end{gather*}

    Il divisore non può essere nullo, quindi $a$ e $b$ non sono nulli.

    Passando alle classi di congruenza, per quanto riguarda il divisore $a$ abbiamo che:
    \begin{gather*}
        a \divisore \overline{ab} \\
        [\overline{ab}]_a = [0]_a \\
        [10]_a[a]_a + [b]_a = [0]_a \\
        [10]_a[0]_a + [b]_a = [0]_a \\
        [b]_a = [0]_a \\
        a \divisore b \\
        \exists k : b = ka
    \end{gather*}

    Invece per quanto riguarda il divisore $b$ abbiamo che:
    \begin{gather*}
        b \divisore \overline{ab} \\
        [\overline{ab}]_b = [0]_b \\
        [10]_b[a]_b + [b]_b = [0]_b \\
        [10]_b[a]_b + [0]_b = [0]_b \\
        [10]_b[a]_b = [0]_b \\
        [10a]_b = [0]_b \\
        b \divisore 10a \\
        ka \divisore 10a \\
        k \divisore 10
    \end{gather*}

    Riassumendo abbiamo ottenuto due condizioni che dobbiamo rispettare:
    \begin{gather*}
        b = ka \\
        k \divisore 10
    \end{gather*}

    I valori possibili di $k$ sono i divisori di 10 che fanno in modo che $b$ sia una cifra.
    Essi sono: 1, 2 e 5.

    Enumeriamo quindi i casi possibili:
    \begin{itemize}
        \item $k = 1 \,\Longrightarrow\, b = a$: ottengo i numeri 11, 22, 33, 44, 55, 66, 77, 88, 99;
        \item $k = 2 \,\Longrightarrow\, b = 2a$: ottengo i numeri 12, 24, 36, 48;
        \item $k = 5 \,\Longrightarrow\, b = 5a$: ottengo il numero 15.
    \end{itemize}

    \bigskip
    \mdfsubtitle{Risposta}
    Gli interi positivi cercati sono: 11, 12, 15, 22, 24, 33, 36, 44, 48, 55, 66, 77, 88, 99.

    \bigskip
    \mdfsubtitle{Soluzione esercizio~\ref{ex:distrettuali_2018}: parte 2)}
    Ora abbiamo a che fare con numeri di due o più cifre.
    Non possiamo più usare la notazione posizionale, ma possiamo comunque scomporre il numero $n$ in $10a + b$,
    dove $a$ è un qualunque numero intero maggiore di 1, e $b$ è una singola cifra.

    Il numero $n = 10a + b$ deve essere divisibile per $a$, quindi:
    \begin{gather*}
        a \divisore 10a + b \\
        [10a + b]_a = [0]_a \\
        [10]_a[a]_a + [b]_a = [0]_a \\
        [10]_a[0]_a + [b]_a = [0]_a \\
        [b]_a = [0]_a \\
        a \divisore b
    \end{gather*}

    Affinché $a$ sia un divisore di $b$ è necessario che $a \le b$.
    Visto che $b$ è una singola cifra, ovvero $b \le 9$, deduciamo che anche $a \le 9$.
    Quindi il numero numero $n = 10a + b$ è composto al più di 2 cifre.
\end{soluzione}

\begin{soluzione}
    \mdfsubtitle{Soluzione esercizio~\ref{ex:distrettuali_2021}}
    I numeri palindromi di 6 cifre hanno la seguente struttura:

    \begin{equation*}
        \overline{abccba}
    \end{equation*}

    Dobbiamo trovare quanti di questi numeri sono divisibili per 33, ovvero sono divisibili per 3 e per 11.

    Innanzitutto osserviamo che tutti i numeri palindromi di 6 cifre sono divisibili per 11, infatti, per il
    criterio di congruenza modulo 11:

    \begin{equation*}
        [\overline{abccba}]_{11} = [-a + b - c + a - b + c]_{11} = [0]_{11}
    \end{equation*}

    Dobbiamo quindi determinare quanti numeri palindromi di 6 cifre sono divisibili per 3.
    Per il criterio di congruenza modulo 3 risulta:
    \begin{align*}
        [\overline{abccba}]_3 &= [a + b + c + c + b + a]_3 = \\
        &= [2a + 2b + 2c]_3 = \\
        &= [2(a+b+c)]_3 = \\
        &= [2]_3[a + b + c]_3 \\
        &= [2]_3[\overline{abc}]_3
    \end{align*}

    Quindi se vogliamo che $\overline{abccba}$ sia divisibile per 3, visto che 2 non lo è, è necessario che $a + b + c$
    sia divisibile per 3.

    A questo punto ti propongo due soluzioni.

    \bigskip
    \textbf{Soluzione velocissima}

    Dobbiamo determinare quanti sono i numeri di tre cifre che sono divisibili per 3.

    I numeri di 3 cifre vanno da 100 a 999 e in totale solo 900.
    Solo uno ogni tre di questi numeri è divisibile per 100, quindi quelli divisibili per 3 sono:

    \begin{equation*}
        900 : 3 = 300
    \end{equation*}

    \bigskip
    \textbf{Soluzione più lunga}

    Continuando con il criterio di congruenza modulo 3 possiamo scrivere:

    \begin{equation*}
        [a + b + c]_3 = [a]_3 + [b]_3 + [c]_3
    \end{equation*}

    Affinché tale numero sia divisibile per 3 dobbiamo scegliere dei valori di $a$, $b$ e $c$ tali che:

    \begin{equation*}
        [a]_3 + [b]_3 + [c]_3 = [0]_3
    \end{equation*}

    Dobbiamo ora andare a contare quante solo le combinazioni di $a$, $b$ e $c$ che soddisfano tale vincolo.
    Anzi: lo
    faremo in due passaggi: prima contiamo le combinazioni delle classi di congruenza $[a]_3$, $[b]_3$ e $[c]_3$ e poi
    da queste passiamo alle cifre.

    Nota che scelta la classe di congruenza per $[a]_3$ e per $[b]_3$, resta automaticamente determinata la classe di
    congruenza per $[c]_3$.
    Per esempio, se $[a]_3 = [1]_3$ e $[b]_3 = [0]$, allora $[c]_3 =  [2]_3$, infatti:

    \begin{equation*}
        [1]_3 + [0]_3 + [2]_3 = [3]_3 = [0]_3
    \end{equation*}

    Allora le possibili combinazioni (considerando le classi di congruenza) sono date dalle 3 possibili scelte di $[a]_3$
    e dalle 3 possibili scelte di $[b]_3$:

    \begin{table}[H]
        \label{tab:distrettuali_2019_1}
        \centering
        \begin{tabular}{ccc}
            \toprule
            $[a]_3$ & $[b]_3$ & $[c]_3$ \\
            \midrule
            $[0]_3$ & $[0]_3$ & $[0]_3$ \\
            $[0]_3$ & $[1]_3$ & $[1]_3$ \\
            $[0]_3$ & $[2]_3$ & $[2]_3$ \\
            $[1]_3$ & $[0]_3$ & $[2]_3$ \\
            $[1]_3$ & $[1]_3$ & $[1]_3$ \\
            $[1]_3$ & $[2]_3$ & $[0]_3$ \\
            $[2]_3$ & $[0]_3$ & $[1]_3$ \\
            $[2]_3$ & $[1]_3$ & $[0]_3$ \\
            $[2]_3$ & $[2]_3$ & $[2]_3$ \\
            \bottomrule
        \end{tabular}
    \end{table}

    Ora sostituiamo le classi di congruenza con la quantità di scelte che abbiamo per ciascuna di esse.
    Ovviamente non conteremo tutti gli infiniti numeri interi che appartengono alle classi di congruenza, ma solo le
    cifre singole.
    Dobbiamo quindi tenere conto che:
    \begin{itemize}
        \item $[0]_3 = \{0, 3, 6, 8\}$ contiene 4 elementi, ma se usata per la prima cifra $[a]_3$ non possiamo
        considerare la cifra 0 altrimenti avremo un numero di sole 5 cifre;
        \item $[1]_3 = \{1, 4, 7\}$ contiene 3 elementi;
        \item $[2]_3 = \{2, 5, 8\}$ contiene 3 elementi.
    \end{itemize}

    Inoltre, per il principio di moltiplicazione, le possibili scelte per ogni riga saranno date dal prodotto delle
    scelte delle singole cifre.
    Quindi:

    \begin{table}[H]
        \label{tab:distrettuali_2019_2}
        \centering
        \begin{tabular}{ccccccc}
            \toprule
            $[a]_3$ & $[b]_3$ & $[c]_3$ & scelte di $[a]_3$ & scelte di $[b]_3$ & scelte di $[c]_3$ & totali \\
            \midrule
            $[0]_3$ & $[0]_3$ & $[0]_3$ & 3 & 4 & 4 & 48 \\
            $[0]_3$ & $[1]_3$ & $[1]_3$ & 3 & 3 & 3 & 27 \\
            $[0]_3$ & $[2]_3$ & $[2]_3$ & 3 & 3 & 3 & 27 \\
            $[1]_3$ & $[0]_3$ & $[2]_3$ & 3 & 4 & 3 & 36 \\
            $[1]_3$ & $[1]_3$ & $[1]_3$ & 3 & 3 & 3 & 27 \\
            $[1]_3$ & $[2]_3$ & $[0]_3$ & 3 & 3 & 4 & 36 \\
            $[2]_3$ & $[0]_3$ & $[1]_3$ & 3 & 4 & 3 & 36 \\
            $[2]_3$ & $[1]_3$ & $[0]_3$ & 3 & 3 & 4 & 36 \\
            $[2]_3$ & $[2]_3$ & $[2]_3$ & 3 & 3 & 3 & 27 \\
            \midrule
            & & & & & & 300 \\
            \bottomrule
        \end{tabular}
    \end{table}

    \bigskip
    \mdfsubtitle{Risposta}
    I numeri palindromi di 6 cifre divisibili per 33 sono 300.
\end{soluzione}

\begin{soluzione}
    \mdfsubtitle{Soluzione esercizio~\ref{ex:francesco_1}}
    Il numero $\overline{abcabc}$ può essere scomposto in:

    \begin{equation*}
        \overline{abcabc} = \overline{abc} \cdot 1001 = \overline{abc} \cdot 7 \cdot 11 \cdot 13
    \end{equation*}

    Il divisore 2023 invece si scompone in:

    \begin{equation*}
        2023 = 7 \cdot 17^2
    \end{equation*}

    Se 2023 divide $\overline{abcabc}$ significa allora che:

    \begin{equation*}
        \exists k \in \Z: \overline{abc} \cdot 7 \cdot 11 \cdot 13 = k \cdot 7 \cdot 17^2
    \end{equation*}

    Possiamo semplificare per 7:

    \begin{equation*}
        \exists k \in \Z: \overline{abc} \cdot 11 \cdot 13 = k \cdot 17^2
    \end{equation*}

    Non abbiamo altri fattori in comune da semplificare, quindi è necessario che $\overline{abc}$ sia divisibile per
    $17^2$.

    Dobbiamo quindi determinare quanti numeri di tre cifre sono divisibili per $17^2 = 289$.

    289 è maggiore di 250, quindi i numeri cercati sono meno di 4:

    \begin{equation*}
        1000 : 289 < 1000 : 250 = 4
    \end{equation*}

    289 è minore di $333 \sim \frac{1000}{3}$, quindi ci sono almeno 3 multipli di 3 cifre:

    \begin{equation*}
        1000 : 289 > 1000 : \frac{1000}{3} = 3
    \end{equation*}

    \bigskip
    \mdfsubtitle{Risposta}
    I numeri nella forma $\overline{abcabc}$ divisibili per 2023 sono 3.

    Essi sono: 289289, 578578 e 867867.

\end{soluzione}
