\chapter{Soluzioni}
\label{ch:soluzioni}

\begin{proof}[Soluzione esercizio~\ref{ex:distrettuali_2019}]
    I numeri palindromi di 6 cifre hanno la seguente struttura:

    \begin{equation*}
        \overline{abccba}
    \end{equation*}

    Dobbiamo trovare quanti di questi numeri sono divisibili per 33, ovvero sono divisibili per 3 e per 11.

    Innanzitutto osserviamo che tutti i numeri palindromi di 6 cifre sono divisibili per 11, infatti, per il
    criterio di congruenza modulo 11:

    \begin{equation*}
        [\overline{abccba}]_{11} = [-a + b - c + a - b + c]_{11} = [0]_{11}
    \end{equation*}

    Dobbiamo quindi determinare quanti numeri palindromi di 6 cifre sono divisibili per 3.
    Per il criterio di congruenza modulo 3 risulta:
    \begin{align*}
        [\overline{abccba}]_3 &= [a + b + c + c + b + a]_3 = \\
        &= [2a + 2b + 2c]_3 = \\
        &= [2(a+b+c)]_3 = \\
        &= [2]_3[a + b + c]_3 \\
        &= [2]_3[\overline{abc}]_3
    \end{align*}

    Quindi se vogliamo che $\overline{abccba}$ sia divisibile per 3, visto che 2 non lo è, è necessario che $a + b + c$
    sia divisibile per 3.

    A questo punto ti propongo due soluzioni.

    \bigskip
    \textbf{Soluzione velocissima}

    Dobbiamo determinare quanti sono i numeri di tre cifre che sono divisibili per 3.

    I numeri di 3 cifre vanno da 100 a 999 e in totale solo 900.
    Solo uno ogni tre di questi numeri è divisibile per 100, quindi quelli divisibili per 3 sono:

    \begin{equation*}
        900 : 3 = 300
    \end{equation*}

    \bigskip
    \textbf{Soluzione più lunga}

    Continuando con il criterio di congruenza modulo 3 possiamo scrivere:

    \begin{equation*}
        [a + b + c]_3 = [a]_3 + [b]_3 + [c]_3
    \end{equation*}

    Affinchè tale numero sia divisibile per 3 dobbiamo scegliere dei valori di $a$, $b$ e $c$ tali che:

    \begin{equation*}
        [a]_3 + [b]_3 + [c]_3 = [0]_3
    \end{equation*}

    Dobbiamo ora andare a contare quante solo le combinazioni di $a$, $b$ e $c$ che soddisfano tale vincolo. Anzi: lo
    faremo in due passaggi: prima contiamo le combinazioni delle classi di congruenza $[a]_3$, $[b]_3$ e $[c]_3$ e poi
    da queste passiamo alle cifre.

    Nota che scelta la classe di congruenza per $[a]_3$ e per $[b]_3$, resta automaticamente determinata la classe di
    congruenza per $[c]_3$.
    Per esempio, se $[a]_3 = [1]_3$ e $[b]_3 = [0]$, allora $[c]_3 =  [2]_3$, infatti:

    \begin{equation*}
        [1]_3 + [0]_3 + [2]_3 = [3]_3 = [0]_3
    \end{equation*}

    Allora le possibili combinazioni (considerando le classi di congruenza) sono date dalle 3 possibili scelte di $[a]_3$
    e dalle 3 possibili scelte di $[b]_3$:

    \begin{table}[H]
        \label{tab:distrettuali_2019_1}
        \centering
        \begin{tabular}{ccc}
            \toprule
            $[a]_3$ & $[b]_3$ & $[c]_3$ \\
            \midrule
            $[0]_3$ & $[0]_3$ & $[0]_3$ \\
            $[0]_3$ & $[1]_3$ & $[1]_3$ \\
            $[0]_3$ & $[2]_3$ & $[2]_3$ \\
            $[1]_3$ & $[0]_3$ & $[2]_3$ \\
            $[1]_3$ & $[1]_3$ & $[1]_3$ \\
            $[1]_3$ & $[2]_3$ & $[0]_3$ \\
            $[2]_3$ & $[0]_3$ & $[1]_3$ \\
            $[2]_3$ & $[1]_3$ & $[0]_3$ \\
            $[2]_3$ & $[2]_3$ & $[2]_3$ \\
            \bottomrule
        \end{tabular}
    \end{table}

    Ora sostituiamo le classi di congruenza con la quantità di scelte che abbiamo per ciascuna di esse.
    Ovviamente non conteremo tutti gli infiniti numeri interi che appartengono alle classi di congruenza, ma solo le
    cifre singole.
    Dobbiamo quindi tenere conto che:
    \begin{itemize}
        \item $[0]_3 = \{0, 3, 6, 8\}$ contiene 4 elementi, ma se usata per la prima cifra $[a]_3$ non possiamo
        considerare la cifra 0 altrimenti avremo un numero di sole 5 cifre;
        \item $[1]_3 = \{1, 4, 7\}$ contiene 3 elementi;
        \item $[2]_3 = \{2, 5, 8\}$ contiene 3 elementi.
    \end{itemize}

    Inoltre, per il principio di moltiplicazione, le possibili scelte per ogni riga saranno date dal prodotto delle
    scelte delle singole cifre.
    Quindi:

    \begin{table}[H]
        \label{tab:distrettuali_2019_2}
        \centering
        \begin{tabular}{ccccccc}
            \toprule
            $[a]_3$ & $[b]_3$ & $[c]_3$ & scelte di $[a]_3$ & scelte di $[b]_3$ & scelte di $[c]_3$ & scelte totali \\
            \midrule
            $[0]_3$ & $[0]_3$ & $[0]_3$ & 3 & 4 & 4 & 48 \\
            $[0]_3$ & $[1]_3$ & $[1]_3$ & 3 & 3 & 3 & 27 \\
            $[0]_3$ & $[2]_3$ & $[2]_3$ & 3 & 3 & 3 & 27 \\
            $[1]_3$ & $[0]_3$ & $[2]_3$ & 3 & 4 & 3 & 36 \\
            $[1]_3$ & $[1]_3$ & $[1]_3$ & 3 & 3 & 3 & 27 \\
            $[1]_3$ & $[2]_3$ & $[0]_3$ & 3 & 3 & 4 & 36 \\
            $[2]_3$ & $[0]_3$ & $[1]_3$ & 3 & 4 & 3 & 36 \\
            $[2]_3$ & $[1]_3$ & $[0]_3$ & 3 & 3 & 4 & 36 \\
            $[2]_3$ & $[2]_3$ & $[2]_3$ & 3 & 3 & 3 & 27 \\
            \midrule
            & & & & & & 300 \\
            \bottomrule
        \end{tabular}
    \end{table}

    \bigskip
    \textbf{Risposta:}
    I numeri palindromi di 6 cifre divisibili per 33 sono 300.
\end{proof}

\begin{proof}[Soluzione esercizio~\ref{ex:francesco_1}]
    Il numero $\overline{abcabc}$ può essere scomposto in:

    \begin{equation*}
        \overline{abcabc} = \overline{abc} \cdot 1001 = \overline{abc} \cdot 7 \cdot 11 \cdot 13
    \end{equation*}

    Il divisore 2023 invece si scompone in:

    \begin{equation*}
        2023 = 7 \cdot 17^2
    \end{equation*}

    Se 2023 divide $\overline{abcabc}$ significa allora che:

    \begin{equation*}
        \exists k \in \Z: \overline{abc} \cdot 7 \cdot 11 \cdot 13 = k \cdot 7 \cdot 17^2
    \end{equation*}

    Possiamo semplificare per 7:

    \begin{equation*}
        \exists k \in \Z: \overline{abc} \cdot 11 \cdot 13 = k \cdot 17^2
    \end{equation*}

    Non abbiamo altri fattori in comune da semplificare, quindi è necessario che $\overline{abc}$ sia divisibile per
    $17^2$.

    Dobbiamo quindi determinare quanti numeri di tre cifre sono divisibili per $17^2 = 289$.

    289 è maggiore di 250, quindi i numeri cercati sono meno di 4:

    \begin{equation*}
        1000 : 289 < 1000 : 250 = 4
    \end{equation*}

    289 è minore di $333 \sim \frac{1000}{3}$, quindi ci sono almeno 3 multipli di 3 cifre:

    \begin{equation*}
        1000 : 289 > 1000 : \frac{1000}{3} = 3
    \end{equation*}

    \bigskip
    \textbf{Risposta:}
    I numeri nella forma $\overline{abcabc}$ divisibili per 2023 sono 3.

    Infatti sono: 289289, 578578 e 867867.

\end{proof}
