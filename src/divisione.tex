\chapter{Divisione con resto}
\label{ch:divisione_con_resto}

\section{Divisione euclidea}
\label{sec:divisione_euclidea}

Fin dalla scuola primaria conosciamo la divisione con resto.
Vogliamo ora vederne la proprietà fondamentale e precisarne la definizione.
Inoltre vogliamo estendere la divisione a tutto l'insieme degli interi $\Z$.

\begin{teorema}
Siano $a, b \in \Z$, con $b \ne 0$.
Allora esistono due numeri interi $q$ e $r$ tali che:

\begin{gather}
    a = qb + r \\
    0 \le r < \valoreassoluto{b}
\end{gather}

Inoltre $q$ ed $r$ sono unici.
\end{teorema}

\begin{definizione}[Divisione euclidea]
L'operazione sopra presentata si chiama \textbf{divisione euclidea}, o \textbf{divisione con resto}. $a$
si chiama \textbf{dividendo}, $b$ si chiama \text{divisore}, $q$ si chiama \textbf{quoziente} e $r$ si chiama \textbf{resto}.
\end{definizione}

Se, con numeri positivi, questa divisione coincide con quella che abbiamo imparato alla scuola primaria, ora abbiamo la novità dell'uso dei numeri negativi.
E l'elemento a cui dobbiamo assolutamente fare attenzione è che il resto \emph{deve sempre essere positivo}.
Analizziamo i singoli casi:

\begin{itemize}
    \item $a < 0 \,\,\land\,\, b > 0$: \\
    Se, per esempio, devo fare la divisione (-14) : 6 non posso usare -2 come quoziente, perché il resto sarebbe negativo:
    \begin{equation*}
        -14 = (-2) \cdot 6 - 2
    \end{equation*}
    Per avere un resto positivo (e minore di 6) devo usare -3 come quoziente:~\footnote{
    Un modo per interpretare questa operazione potrebbe essere il seguente problema:
    
    \emph{Per fare una torta alla crema servono 14 uova, ma le confezioni sono da 6 uova ciascuna. Quante confezioni di uova devo prendere?} 
    
    Non posso dire che devo prendere 2 confezioni, perché mi mancherebbero altre 2 uova. 
    
    Devo prendere 3 confezioni e mi avanzeranno 4 uova.}
    \begin{equation*}
        -14 = (-3) \cdot 6 + 4
    \end{equation*}

    \item $a > 0 \,\,\land\,\, b < 0$: \\
    Se, per esempio, devo fare la divisione 14 : (-6) devo usare -2 come quoziente, perché il resto è positivo:
    \begin{equation*}
        14 = (-2) \cdot (-6) + 2
    \end{equation*}

    \item $a < 0 \,\,\land\,\, b < 0$: \\
    Se, per esempio, devo fare la divisione (-14) : (-6) devo usare 3 come quoziente:
    \begin{equation*}
        -14 = (3) \cdot (-6) + 4
    \end{equation*}
\end{itemize}

\section{L'operazione "modulo"}
\label{sec:modulo}

In molte situazioni ci interessa solo il resto di una divisione, per cui introduciamo l'operazione di \emph{modulo}:

\begin{definizione}[Modulo]
    L'operazione \textbf{modulo} $\bmod$ restituisce il solo resto di una divisione euclidea:
    \begin{equation*}
        a \bmod b = r \sse \exists q : a = qb + r \text{ con } 0 \le r < \valoreassoluto{b}
    \end{equation*}
\end{definizione}

Esempi:
\begin{align*}
    14 \bmod 6 = 2 &\quad\quad\text{perché } 14 = 2\cdot 6 + 2 \\
    -14 \bmod 6 = 4 &\quad\quad\text{perché } -14 = -3 \cdot 6 + 4
\end{align*}

\section{La relazione divide}
\label{sec:divide}

\begin{definizione}[Divide]
    Nel caso in cui
    \begin{equation*}
        a \modulo b = 0
    \end{equation*}
    
    si dice che il $b$ \textbf{divide} $a$ e si scrive:
    \begin{equation*}
        b \divisore a
    \end{equation*}

    Quindi:
    \begin{equation*}
        b \divisore a \sse a \modulo b = 0 \sse \exists q : a = qb
    \end{equation*}
    
\end{definizione}

\begin{teorema}
    \begin{equation*}
        b \divisore a \sse bp \divisore ap
    \end{equation*}

    \mdfsubtitle{Dimostrazione}

    Basta far riferimento alla seconda legge di monotonia:
    \begin{equation*}
        b \divisore a \sse \exists q : a = qb \sse \exists q : ap = qbp \sse bp \divisore ap
    \end{equation*}
\end{teorema}

\section{Esercizi}
\label{sec:esercizi_divisione}

\textbf{Esercizio 1}
Per ognuna delle seguenti affermazioni dire se è vera o falsa;
$p$ denota un numero primo, $a$ e $b$ generici numeri interi.
\begin{gather*}
    p \divisore ab \allora p \divisore a \lor p \divisore b \\
    p^2 \divisore ab \allora p^2 \divisore a \lor p^2 \divisore b \\
    p \divisore a-b \allora p \divisore a+b \\
    p^2 \divisore a^3 \allora p^6 \divisore a^6 \\
    p^2 \divisore b^3 \allora p^2 \divisore b
\end{gather*}