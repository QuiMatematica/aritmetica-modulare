\chapter{Aritmetica modulare}

In questo capitolo $m$ indica un numero intero > 1.

\section{Congruenza modulo $m$}

\begin{mdframed}
    \begin{definizione}
        Due numeri interi $a$ e $b$ sono \textbf{congruenti modulo $m$}, e si scrive:
    
        \begin{equation}
            \congruenti{a}{b}{m}
        \end{equation}
        
        se, divisi per $m$, danno lo stesso resto:
    
        \begin{equation}
            \label{eq:def_congruenza} \congruenti{a}{b}{m} \sse a \modulo m = b \modulo m
        \end{equation}
    
    \end{definizione}
\end{mdframed}

Per esempio, tornando al quadrante dell'orologio, possiamo dire che 2 e 14 sono \emph{congruenti modulo 12}, in quanto:

\begin{equation*}
    \begin{cases}
        2 \modulo 12 = 2 \\
        14 \modulo 12 = 2
    \end{cases}
    \quad\Longrightarrow\quad
    \congruenti{2}{14}{12}
\end{equation*}

Se $m = 3$, allora i numeri 22 e 67 sono \emph{congruenti modulo 3}, quanto:

\begin{equation*}
    \begin{cases}
        22 \modulo 3 = 1 \\
        67 \modulo 3 = 1
    \end{cases}
    \quad\Longrightarrow\quad
    \congruenti{22}{67}{3}
\end{equation*}

\begin{mdframed}
    \begin{teorema}
        \label{th:congruenza}
        \begin{equation}
            \label{eq:th_congruenza} \congruenti{a}{b}{m} \sse m \divisore (a - b)
        \end{equation}
    \end{teorema}
    \begin{proof}[Dimostrazione $\Longrightarrow$]
        Faccio la divisione euclidea di $a$ e $m$ e ottengo un quoziente $q_a$ e un resto $r$:
        \begin{equation*}
            a = q_a m + r
        \end{equation*}

        Faccio la divisione euclidea di $b$ e $m$ e ottengo un quoziente $q_b$ e lo stesso resto $r$ in quanto $a$ e $b$ sono congruenti modulo $m$ per ipotesi:
        \begin{equation*}
            b = q_b m + r
        \end{equation*}

        Faccio ora la differenza tra $a$ e $b$ e ottengo:
        \begin{align*}
            a - b &= (q_a m + r) - (q_b m + r) = \\
            &= q_a m + r - q_b m - r = \\
            &= q_a m - q_b m = \\
            &= (q_a - q_b)m
        \end{align*}

        Quindi la divisione euclidea tra $a - b$ e $m$ dà resto 0, quindi $m$ divide la differenza $a - b$:
        \begin{equation*}
            m \divisore (a - b)
        \end{equation*}
    \end{proof}
\end{mdframed}

Per esempio, nella matematica dell'orologio, 45 e 21 sono congruenti modulo 12 perché 12 divide 45 - 21

\begin{equation*}
    \congruenti{45}{21}{12} \sse 12 \divisore (45 - 21)
\end{equation*}

Dal momento che le formule~\eqref{eq:def_congruenza} e~\eqref{eq:th_congruenza} sono delle equivalenze logiche, potrei usare la~\eqref{eq:th_congruenza} come definizione di congruenza, e dimostrare la~\eqref{eq:def_congruenza} come teorema.

\begin{mdframed}
    \begin{teorema}
        \label{th:divisore_della_differenza}
        \begin{equation}
            m \divisore a \quad\land\quad m \divisore b \allora m \divisore (a - b)
        \end{equation}
    \end{teorema}
    \begin{proof}
        Se $m$ divide $a$, significa che il resto della divisione $a : m$ è nullo. Lo stesso possiamo dire per $b$. Quindi dividendo $a$ e $b$ per $m$ otteniamo lo stesso resto, quindi i due numeri sono congruenti modulo $m$:
        \begin{equation}
            m \divisore a \quad\land\quad m \divisore b \allora \congruenti{a}{b}{m}
        \end{equation}

        Per il teorema~\ref{th:congruenza} possiamo inoltre affermare:
        \begin{equation}
            m \divisore a \quad\land\quad m \divisore b \allora \congruenti{a}{b}{m} \allora m \divisore (a-b)
        \end{equation}
    \end{proof}
\end{mdframed}

\begin{mdframed}
    \begin{teorema}
        La congruenza modulo $m$ è una relazione di equivalenza.
    \end{teorema}
    \begin{proof}
        Una relazione è di equivalenza se gode delle proprietà riflessiva, simmetrica e transitiva.

        La congruenza gode della proprietà \emph{riflessiva} infatti per la proprietà riflessiva dell'uguaglianza:
        \begin{equation*}
            \congruenti{a}{a}{m} \sse a \modulo m = a \modulo m
        \end{equation*}
        
        La congruenza gode della proprietà \emph{simmetrica} infatti:
        \begin{equation*}            
            \congruenti{a}{b}{m} \allora a \modulo m = b \modulo m 
        \end{equation*}
        quindi per la proprietà simmetrica dell'uguaglianza:
        \begin{equation*}
            \allora b \modulo m = a \modulo m \allora \congruenti{b}{a}{m}
        \end{equation*}
        
        La congruenza gode della proprietà \emph{transitiva} infatti:
        \begin{gather*}
            \congruenti{a}{b}{m} \allora a \modulo m = b \modulo m \\
            \congruenti{b}{c}{m} \allora b \modulo m = c \modulo m
        \end{gather*}
        quindi per la proprietà transitiva dell'uguaglianza:
        \begin{equation*}
            \allora a \modulo m = c \modulo m \allora \congruenti{a}{c}{m}
        \end{equation*}
    \end{proof}
\end{mdframed}

\section{Le classi di congruenza}

Come mostrato nel caso dell'orologio, la congruenza modulo $m$ divide l'insieme $\Z$ dei numeri interi in $m$ sottoinsiemi. Per esempio, se consideriamo la congruenza modulo 3, abbiamo i seguenti insiemi:
\begin{gather*}
    \{\dots, -6, -3, 0, 3, 6, \dots\} \\
    \{\dots, -5, -2, 1, 4, 7, \dots\} \\
    \{\dots, -4, -1, 2, 5, 8, \dots\}
\end{gather*}

Questi insiemi hanno due caratteristiche significative:
\begin{itemize}
    \item ogni numero intero appartiene ad uno di questi insiemi, ovvero l'unione degli insiemi forma l'insieme $\Z$;
    \item ogni numero intero appartiene ad uno solo di questi insiemi, ovvero gli insiemi sono tra loro disgiunti.
\end{itemize}

Quando si presentano queste due caratteristiche si dice che gli insiemi formano una \textbf{partizione} dell'insieme di partenza, quindi possiamo dire che i tre insiemi sopra presentati \emph{partizionano} l'insieme degli interi $\Z$.

\begin{mdframed}
    \begin{definizione}
        Gli insiemi che formano una partizione degli interi in base ad una congruenza modulo $m$ si chiamano \textbf{classi di congruenza}.
    \end{definizione}
\end{mdframed}

Per indicare la classe di congruenza a cui appartiene un interno $a$, relativamente al modulo $m$, si utilizza la seguente notazione:

\begin{equation*}
    [a]_m
\end{equation*}

Talvolta, quando si lavora con un unico modulo o quando è chiaro quale modulo si sta usando, si può omettere il pedice $m$.

Tale insieme conterrà tutti i numeri interi che sono congruenti modulo $m$ dell'elemento $a$ scelto, per cui possiamo scrivere:
\begin{equation*}
    [a]_m = \{z \in \Z \mid \congruenti{z}{a}{m}\}
\end{equation*}

Tra tutti gli elementi di una classe di congruenza consideriamo rappresentante privilegiato quello positivo e minore di $m$. Per cui i rappresentanti delle classi di congruenza a modulo 3 sono:

\begin{equation*}
    [0]_3, [1]_3, [2]_3
\end{equation*}

Quando vorremo riferirci alle classi di congruenza useremo normalmente questi rappresentanti privilegiati.

E' immediato notare che i rappresentanti privilegiati coincidono con i resti della divisione per $m$ di tutti gli elementi di ciascuna classe di congruenza. Per cui vale anche questa definizione:
\begin{equation*}
    [a]_m = \{z \in \Z \mid z \modulo m = a\} \quad \text{con } 0 \le a < m
\end{equation*}

Le classi di congruenza formano un nuovo insieme che indicheremo con $\Z_m$:
\begin{equation*}
    \Z_m = \{ [a]_m \mid 0 \le a < m \}
\end{equation*}

Per esempio:
\begin{gather*}
    \Z_2 = \{[0]_2, [1]_2\} \\
    \Z_3 = \{[0]_3, [1]_3, [2]_3\} \\
    \Z_4 = \{[0]_4, [1]_4, [2]_4, [3]_4\} \\
\end{gather*}

\section{Operazioni tra classi di congruenza}

Quello che vogliamo scoprire, ora, è se è possibile inventare una nuova aritmetica che lavori con la classi di congruenza.\footnote{Non dovrebbe stupirti questo programma di costruire una nuova aritmetica basata sulle classi di congruenza. Dopotutto anche le frazioni sono della classi di congruenza, ricordi? In particolare:
\begin{equation*}
    \dfrac{a}{b} = \{ (x, y) \in \Z^2 \mid \exists p \in \Z, p \ne 0: x = pa; y = pb\}
\end{equation*}
Per ciascuna coppia ordinata il primo numero è un numeratore, mentre il secondo numero è un denominatore.

Per esempio:
\begin{equation*}
    \dfrac{2}{3} = \{ \dots, (-4, -6), (-2,-3), (2, 3), (4, 6), (6, 9), \dots \}
\end{equation*}
}

\begin{mdframed}
    \begin{teorema}
        Se
        \begin{equation*}
            \congruenti{a_1}{a_2}{m} \quad\land\quad \congruenti{b_1}{b_2}{m}
        \end{equation*}
        allora:
        \begin{gather*}
            \congruenti{a_1 + b_1}{a_2 + b_2}{m} \\
            \congruenti{a_1 \cdot b_1}{a_2 \cdot b_2}{m} \\
            \congruenti{a_1 - b_1}{a_2 - b_2}{m} 
        \end{gather*}
    \end{teorema}
    \begin{proof}[Dimostrazione della somma]
        Se $a_1$ e $a_2$ sono congruenti, allora le divisioni hanno lo stesso resto, che chiamo $r_a$;
        similmente chiamo $r_b$ il resto delle divisioni di $b_1$ e $b_2$
        \begin{gather*}
            a_1 = q_{a_1} m + r_a \\
            a_2 = q_{a_2} m + r_a \\
            b_1 = q_{b_1} m + r_b \\
            b_2 = q_{b_2} m + r_b 
        \end{gather*}
        Calcoliamo ora le somme che ci interessano:
        \begin{gather*}
            a_1 + b_1 = q_{a_1} m + r_a + q_{b_1} m + r_b = (q_{a_1} + q_{b_1}) m + (r_a + r_b) \\
            a_2 + b_2 = q_{a_2} m + r_a + q_{b_2} m + r_b = (q_{a_2} + q_{b_2}) m + (r_a + r_b) 
        \end{gather*}
        Risulta quindi che le divisioni di $a_1 + b_1$ e di $a_2 + b_2$ per $m$ hanno entrambi il resto uguale a $r_a + r_b$. Quindi le due somme sono congruenti.
    \end{proof}
\end{mdframed}

Questo teorema ha una conseguenza importantissima. Prendiamo due classi di congruenza: $[a]_m$ e $[b]_m$. Qualunque sia l'elemento che prendo della prima classe e qualunque sia l'elemento che prendo dalla seconda classe, la loro somma apparterrà sempre alla medesima classe.

Per esempio, qualsiasi elemento prenda da $[1]_4$ e qualsiasi elemento prenda da $[2]_4$, la loro somma apparterrà sempre alla classe $[3]_4$. 

Questo mi permette di introdurre una nuova operazione.

\begin{mdframed}
    \begin{definizione}
        Definisco la somma tra due classi di equivalenza come la classe di equivalenza che contiene la somma di due loro rappresentanti:

        \begin{equation}
            \label{eq:somma} [a]_m + [b]_m = [a + b]_m
        \end{equation}
    \end{definizione}
\end{mdframed}

Similmente:

\begin{mdframed}
    \begin{definizione}
        Definisco il prodotto tra due classi di equivalenza come la classe di equivalenza che contiene il prodotto di due loro rappresentanti:

        \begin{equation}
            \label{eq:prodotto} [a]_m \cdot [b]_m = [a \cdot b]_m
        \end{equation}
    \end{definizione}
\end{mdframed}

\begin{mdframed}
    \begin{definizione}
        Definisco la differenza tra due classi di equivalenza come la classe di equivalenza che contiene la differenza di due loro rappresentanti:

        \begin{equation}
            \label{eq:differenza} [a]_m - [b]_m = [a - b]_m
        \end{equation}
    \end{definizione}
\end{mdframed}

Queste nuove operazioni mi permettono di semplificare i calcoli quando voglio calcolare i resti delle divisioni.

\begin{mdframed}
    \begin{esercizio}
        Calcola il seguente resto:

        \begin{equation*}
            (73 \cdot 26 + 94) \modulo 7
        \end{equation*}
    \end{esercizio}    
    \begin{proof}[Soluzione]
        Calcolare il resto richiesto corrisponde a determinare qual è la classe di congruenza del risultato dell'operazione al dividendo:
        \begin{align*}
            [73 \cdot 26 + 94]_7 &= [73 \cdot 26]_7 + [94]_7 &&\text{per la~\eqref{eq:somma}} \\
            &= [73]_7 \cdot [26]_7 + [94]_7 &&\text{per la~\eqref{eq:prodotto}} \\
            &= [3]_7 \cdot [5]_7 + [3]_7 &&\text{per congruenza modulo 7} \\
            &= [3 \cdot 5]_7 + [3]_7 &&\text{per la~\eqref{eq:prodotto}} \\
            &= [15]_7 + [3]_7 \\
            &= [15 + 3]_7 &&\text{per la~\eqref{eq:somma}} \\
            &= [18]_7 \\
            &= [4]_7 &&\text{per congruenza modulo 7}
        \end{align*}

        Quindi:
        \begin{equation*}
            (73 \cdot 26 + 94) \modulo 7 = 4
        \end{equation*}
    \end{proof}
\end{mdframed}
