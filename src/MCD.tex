\chapter{Massimo Comun Divisore}
\label{ch:mcd}

Introduciamo una notazione alternativa per il MCD di due numeri (si usa solo con due numeri, non con tre o più):
\begin{equation*}
    (a, b) = MCD(a, b)
\end{equation*}

\section{Metodo delle sottrazioni successive}
\label{sec:metodo_sottrazioni_successive}

Il teorema~\ref{th:divisore_della_differenza} ci permette d'introdurre una tecnica di calcolo del MCD alternativa a quella tipicamente insegnata a scuola (scomposizione e ricerca dei fattori comuni).

Dati due numeri $a$ e $b$, l'MCD è il massimo dei numeri che contiene divisori di entrambi i numeri.
Ma il teorema ci dice che se i due numeri hanno uno stesso divisore $m$, tale numero divide anche la differenza $a - b$.

\begin{definizione}[Metodo delle sottrazioni successive]

Possiamo calcolare l'MCD di due numeri sostituendo il maggiore con la differenza dei due.
E poi reiteriamo questa sostituzione fino a quando i due numeri residui sono uguali.

\end{definizione}

\begin{esercizio}
    Calcolare il MCD tra 6120 e 3780.

    \mdfsubtitle{Soluzione}
    \begin{align*}
        (6120, 3780) &= (2340, 3780) &&\text{perché } 6120 - 3780 = 2340 \\
        &= (2340, 1440) &&\text{perché } 3780 - 2340 = 1440 \\
        &= (900, 1440) &&\text{perché } 2340 - 1440 = 900 \\
        &= (900, 540) &&\text{perché } 1440 - 900 = 540 \\
        &= (360, 540) &&\text{perché } 900 - 540 = 360 \\
        &= (360, 180) &&\text{perché } 540 - 360 = 180 \\
        &= (180, 180) &&\text{perché } 360 - 180 = 180 \\
        &= 180
    \end{align*}
\end{esercizio}

Se poi dobbiamo calcolare anche il mcm, ci basta ricordare la relazione fondamentale:
\begin{equation*}
    a \cdot b = MCD(a, b) \cdot mcm(a, b)
\end{equation*}

\section{Esercizi}
\label{sec:esercizi_mcd}

\textbf{Esercizio 1}
Dire se la seguente affermazione è vera o falsa;
$p$ denota un numero primo, $a$ e $b$ generici numeri interi.
\begin{equation*}
    p \divisore (a, b) \allora p \divisore (a^2, ab)
\end{equation*}