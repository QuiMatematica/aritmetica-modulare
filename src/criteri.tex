\chapter{Criteri di congruenza e di divisibilità}
\label{ch:criteri_congruenza}

In questo capitolo avremo spesso a che fare con le cifre di cui è composto un numero.
E avremo bisogno di sostituire le singole cifre con delle lettere.
Mettiamo una riga sopra ad una rappresentazione di un numero per indicare la giustapposizione delle cifre e non il prodotto dei numeri.
Per esempio:
\begin{equation*}
    a=1, b=2, c=3 \allora \overline{abc} = 123.
\end{equation*}

Dobbiamo tenere presente, inoltre, la scomposizione dei numeri in unità, decine, centinaia, eccetera.
Quindi:
\begin{equation*}
    \overline{abc} = a \cdot 100 + b \cdot 10 + c
\end{equation*}

\section{Criteri di congruenza}
\label{sec:criteri_congruenza}

I criteri di congruenza servono per determinare a che cosa è congruente un intero fissato.

\begin{teorema}
    Un intero è congruente modulo 2 alla sua cifra delle unità.

    \mdfsubtitle{Dimostrazione}
    Presento la dimostrazione per numeri di 3 cifre.
    La dimostrazione è facilmente estendibile ad un qualunque numero di cifre.

    Consideriamo il numero $\overline{abc}$.
    Risulta:
    \begin{align*}
        [\overline{abc}]_2 &= [a \cdot 100 + b \cdot 10 + c]_2 = \\
        &= [a]_2 \cdot [100]_2 + [b]_2 \cdot [10]_2 + [c]_2 = \\
        &= [a]_2 \cdot [0]_2 + [b]_2 \cdot [0]_2 + [c]_2 = \\
        &= [0]_2 + [0]_2 + [c]_2 = \\
        &= [c]_2
    \end{align*}
\end{teorema}

\begin{teorema}
    Un intero è congruente modulo 3 alla somma delle sue cifre.

    \mdfsubtitle{Dimostrazione}
    Presento la dimostrazione per numeri di 3 cifre.
    La dimostrazione è facilmente estendibile ad un qualunque numero di cifre.

    Consideriamo il numero $\overline{abc}$.
    Risulta:
    \begin{align*}
        [\overline{abc}]_3 &= [a \cdot 100 + b \cdot 10 + c]_3 = \\
        &= [a]_3 \cdot [100]_3 + [b]_2 \cdot [10]_3 + [c]_3 = \\
        &= [a]_3 \cdot [1]_3 + [b]_3 \cdot [1]_3 + [c]_3 = \\
        &= [a]_3 + [b]_3 + [c]_3 = \\
        &= [a + b + c]_3
    \end{align*}
\end{teorema}

In sintesi:

\begin{table}[H]
    \begin{mdframed}    
        \label{tab:criteri_congruenza}
        \centering
        \begin{tabular}{c|rcl}
            \toprule
            Modulo & Criterio & di & congruenza \\
            \midrule
            modulo 2 & $[\overline{abcde}]_2 $ & $ = $ & $ [e]_2$ \\
            modulo 3 & $[\overline{abcde}]_3 $ & $ = $ & $ [a + b + c + d + e]_3$ \\
            modulo 4 & $[\overline{abcde}]_4 $ & $ = $ & $ [\overline{de}]_4$ \\
            modulo 5 & $[\overline{abcde}]_5 $ & $ = $ & $ [e]_5$ \\
            modulo 8 & $[\overline{abcde}]_8 $ & $ = $ & $ [\overline{cde}]_8$ \\
            modulo 9 & $[\overline{abcde}]_9 $ & $ = $ & $ [a + b + c + d + e]_9$ \\
            modulo 10 & $[\overline{abcde}]_{10} $ & $ = $ & $ [e]_{10}$ \\
            modulo 11 & $[\overline{abcde}]_{11} $ & $ = $ & $ [a - b + c - d + e]_{11}$ \\
            \bottomrule
        \end{tabular}
        \caption{Criteri di congruenza}
    \end{mdframed}
\end{table}

\section{Criteri di divisibilità}
\label{sec:criteri_divisibilita}

Dai criteri di congruenza derivano i criteri di divisibilità.

\begin{teorema}
    Un intero è divisibile per 2 se e solo se è pari la cifra delle unità.

    \mdfsubtitle{Dimostrazione}
    Per il criterio di congruenza modulo 2:
    \begin{equation*}
        [\overline{abcde}]_2 = [e]_2
    \end{equation*}
    Affinché il numero $\overline{abcde}$ sia divisibile per 2, deve essere congruente a 0, ovvero il numero deve appartenere alla classe di congruenza dello 0:
    \begin{equation*}
        [\overline{abcde}]_2 = [e]_2 = [0]_2
    \end{equation*}
    Da questo deduciamo che un numero è divisibile per 2 se la sua cifra delle unità è divisibile per 2, ovvero è pari.
\end{teorema}

\begin{teorema}
    Un intero è divisibile per 3 se e solo se la somma delle sue cifre è divisibile per 3.

    \mdfsubtitle{Dimostrazione:}
    Per il criterio di congruenza modulo 3:
    \begin{equation*}
        [\overline{abcde}]_3 = [a + b + c + d + e]_3
    \end{equation*}
    Affinché il numero $\overline{abcde}$ sia divisibile per 3, deve essere congruente a 0, ovvero il numero deve appartenere alla classe di congruenza dello 0:
    \begin{equation*}
        [\overline{abcde}]_3 = [a + b + c + d + e]_3 = [0]_2
    \end{equation*}
    Da questo deduciamo che un numero è divisibile per 3 se la somma delle sue cifre è divisibile per 3.
\end{teorema}

Le dimostrazioni dei rimanenti criteri di visibilità sono identiche a queste due presentate, e conducono ai criteri di divisibilità che già conosci.

Aggiungiamo un criterio di divisibilità che non derivano da un criterio di congruenza:

\begin{teorema}
    Un intero è divisibile per 7 se e solo se è divisibile per 7 l'intero costituito nel seguente modo: si prende l'intero iniziale privato della cifra delle unità e gli si sottrae il doppio della cifra delle unità.

    \mdfsubtitle{Dimostrazione parte se}
    Per semplicità presentiamo la dimostrazione per numeri di due cifre.
    La dimostrazione è facilmente estendibile a numeri di qualunque lunghezza.

    Quello che vogliamo dimostrare è che il numero $\overline{ab}$ è divisibile per 7 se è divisibile il numero costruito prendendo l'intero iniziale privato della cifra delle unità (e quindi rimane solo la cifra delle decine) a cui si sottrae il doppio della cifra delle unità, ovvero se è divisibile per 7 il numero $a - 2b$.
    Quindi, passando alle classi di congruenza, dobbiamo dimostrare che:
    \begin{equation*}
        [a - 2b]_7 = [0]_7 \allora [\overline{ab}]_7 = [0]_7
    \end{equation*}
    Dall'ipotesi con cui stiamo lavorando possiamo ricavare che:
    \begin{equation*}
        [a]_7 = [2b]_7
    \end{equation*}
    Calcoliamo allora la classe di congruenza del numero $\overline{ab}$:
    \begin{align*}
        [\overline{ab}]_7 &= [a \cdot 10 + b]_7 = \\
        &= [a]_7 \cdot [10]_7 + [b]_7 =\\
        &= [2b]_7 \cdot [3]_7 + [b]_7 =\\
        &= [6b]_7 + [b]_7 =\\
        &= [7b]_7 =\\
        &= [0]_7
    \end{align*}

\end{teorema}



