\documentclass[a4paper,12pt]{book}
\usepackage[utf8]{inputenc}
\usepackage[T1]{fontenc}
\usepackage[italian]{babel}
\usepackage{booktabs}
\usepackage{graphicx}
\usepackage{amsmath}
\usepackage{amssymb}
\usepackage{siunitx}
\usepackage[framemethod=TikZ]{mdframed}
\usepackage{float}

\mdfdefinestyle{definitionstyle}{%
	linecolor=green,linewidth=2pt,%
	frametitlerule=true,%
	frametitlebackgroundcolor=green!20,%
	innertopmargin=\topskip,
}

\mdfdefinestyle{theoremstyle}{%
	linecolor=red,linewidth=2pt,%
	frametitlerule=true,%
	frametitlebackgroundcolor=red!20,%
	innertopmargin=\topskip,
}

\mdfdefinestyle{exercisestyle}{%
	linecolor=blue,linewidth=2pt,%
	frametitlerule=true,%
	frametitlebackgroundcolor=blue!20,%
	innertopmargin=\topskip,
}

\mdtheorem[style=definitionstyle]{definizione}{Definizione}[chapter]
\mdtheorem[style=theoremstyle]{teorema}{Teorema}[chapter]
\mdtheorem[style=exercisestyle]{esercizio}{Esercizio}[chapter]

\newmdenv[linecolor=red,linewidth=2pt]{dimostrazione}
\newmdenv[linecolor=blue,linewidth=2pt]{soluzione}

\newcommand{\Z}[0]{\ensuremath{ \mathbb{Z} }}
\newcommand{\valoreassoluto}[1]{\ensuremath{ \lvert #1 \rvert }}
\newcommand{\divisore}[0]{\ensuremath{ \mid }}
\newcommand{\sse}[0]{\ensuremath{ \quad\Longleftrightarrow\quad }}
\newcommand{\allora}[0]{\ensuremath{ \quad\Longrightarrow\quad }}
\newcommand{\modulo}[0]{\ensuremath{ \bmod }}
\newcommand{\congruenti}[3]{\ensuremath{ #1 \equiv #2 \quad\bmod #3 }}

\begin{document}

\begin{titlepage}
    \begin{center}
        \vspace*{1cm}
        
        \LARGE
        Istituto A. Einstein - Piove di Sacco
        
        \vspace{1.5cm}
        
        \Huge
        \textbf{Introduzione all'aritmetica modulare}
        
        \vspace{1.5cm}
                    
        \LARGE
        \today
            
        \vfill
        
    \end{center}
\end{titlepage}
\chapter{Divisione con resto}
\label{ch:divisione_con_resto}

\section{Divisione euclidea}
\label{sec:divisione_euclidea}

Fin dalla scuola primaria conosciamo la divisione con resto.
Vogliamo ora vederne la proprietà fondamentale e precisarne la definizione.
Inoltre vogliamo estendere la divisione a tutto l'insieme degli interi $\Z$.

\begin{teorema}
Siano $a, b \in \Z$, con $b \ne 0$.
Allora esistono due numeri interi $q$ e $r$ tali che:

\begin{gather}
    a = qb + r \\
    0 \le r < \valoreassoluto{b}
\end{gather}

Inoltre $q$ ed $r$ sono unici.
\end{teorema}

\begin{definizione}[Divisione euclidea]
L'operazione sopra presentata si chiama \textbf{divisione euclidea}, o \textbf{divisione con resto}. $a$
si chiama \textbf{dividendo}, $b$ si chiama \text{divisore}, $q$ si chiama \textbf{quoziente} e $r$ si chiama \textbf{resto}.
\end{definizione}

Se, con numeri positivi, questa divisione coincide con quella che abbiamo imparato alla scuola primaria, ora abbiamo la novità dell'uso dei numeri negativi.
E l'elemento a cui dobbiamo assolutamente fare attenzione è che il resto \emph{deve sempre essere positivo}.
Analizziamo i singoli casi:

\begin{itemize}
    \item $a < 0 \,\,\land\,\, b > 0$: \\
    Se, per esempio, devo fare la divisione (-14) : 6 non posso usare -2 come quoziente, perché il resto sarebbe negativo:
    \begin{equation*}
        -14 = (-2) \cdot 6 - 2
    \end{equation*}
    Per avere un resto positivo (e minore di 6) devo usare -3 come quoziente:~\footnote{
    Un modo per interpretare questa operazione potrebbe essere il seguente problema:
    
    \emph{Per fare una torta alla crema servono 14 uova, ma le confezioni sono da 6 uova ciascuna. Quante confezioni di uova devo prendere?} 
    
    Non posso dire che devo prendere 2 confezioni, perché mi mancherebbero altre 2 uova. 
    
    Devo prendere 3 confezioni e mi avanzeranno 4 uova.}
    \begin{equation*}
        -14 = (-3) \cdot 6 + 4
    \end{equation*}

    \item $a > 0 \,\,\land\,\, b < 0$: \\
    Se, per esempio, devo fare la divisione 14 : (-6) devo usare -2 come quoziente, perché il resto è positivo:
    \begin{equation*}
        14 = (-2) \cdot (-6) + 2
    \end{equation*}

    \item $a < 0 \,\,\land\,\, b < 0$: \\
    Se, per esempio, devo fare la divisione (-14) : (-6) devo usare 3 come quoziente:
    \begin{equation*}
        -14 = (3) \cdot (-6) + 4
    \end{equation*}
\end{itemize}

\section{L'operazione "modulo"}
\label{sec:modulo}

In molte situazioni ci interessa solo il resto di una divisione, per cui introduciamo l'operazione di \emph{modulo}:

\begin{definizione}[Modulo]
    L'operazione \textbf{modulo} $\bmod$ restituisce il solo resto di una divisione euclidea:
    \begin{equation*}
        a \bmod b = r \sse \exists q : a = qb + r \text{ con } 0 \le r < \valoreassoluto{b}
    \end{equation*}
\end{definizione}

Esempi:
\begin{align*}
    14 \bmod 6 = 2 &\quad\quad\text{perché } 14 = 2\cdot 6 + 2 \\
    -14 \bmod 6 = 4 &\quad\quad\text{perché } -14 = -3 \cdot 6 + 4
\end{align*}

\section{La relazione divide}
\label{sec:divide}

\begin{definizione}[Divide]
    Nel caso in cui
    \begin{equation*}
        a \modulo b = 0
    \end{equation*}
    
    si dice che il $b$ \textbf{divide} $a$ e si scrive:
    \begin{equation*}
        b \divisore a
    \end{equation*}

    Quindi:
    \begin{equation*}
        b \divisore a \sse a \modulo b = 0 \sse \exists q : a = qb
    \end{equation*}
    
\end{definizione}


\chapter{La matematica dell'orologio}

Consideriamo un orologio analogico, e consideriamo la sola lancetta delle ore. I numeri mostrati sul quadrante corrispondono alle ore.

In questi esercizi trascuriamo la differenza tra ore antimeridiane e ore pomeridiane: vogliamo concentrarci solo al numero che definisce l'ora e al numero di giri che compie la lancetta delle ore.

Inoltre, visto che avremo a che fare solo con numeri interi, consideriamo solo ore "intere", con la lancetta che punta precisamente ad un numero, senza considerare tutte le posizioni intermedie. 

    \begin{esercizio}
        A quanti giri della lancetta delle ore corrispondono 120 ore?
    \end{esercizio}
    \begin{soluzione}
        Visto che un giro della lancetta delle ore corrisponde a 12 ore, in 120 ore la lancetta farà:
        
        \begin{equation*}
            \text{giri} = (\text{ore}) : (\text{ore per giro}) = 120 : 12 = 10
        \end{equation*}
        
        \emph{La lancetta compie 10 giri.}
    \end{soluzione}

\begin{esercizio}
    Sono le 12. Che ore saranno tra 125 ore?
\end{esercizio}
\begin{soluzione}
    In questo caso non ci interessano i numeri di giri, ma ci interessa solo cosa succede dopo l'ultimo giro completo. Quello che ci interessa, quindi è il resto della divisione:
    
    \begin{equation*}
        \text{ore future} = 125 \modulo 12 = 5
    \end{equation*}
    
    \emph{Dopo 125 ore saranno le 5.}
\end{soluzione}

    \begin{esercizio}
        Sono le 5. Che ore saranno tra 21 ore?
    \end{esercizio}
    \begin{soluzione}
        \emph{Soluzione 1:}
        Togliamo i giri inutili:

        \begin{equation*}
            \text{ore da aggiungere} = 21 \modulo 12 = 9
        \end{equation*}

        Aggiungiamo queste ore alle ore attuali:

        \begin{equation*}
            \text{ore future} = 5 + 9 = 14
        \end{equation*}

        Ma sull'orologio non ho le ore 14, per cui devo togliere il giro inutile:

        \begin{equation*}
            \text{ore normalizzate} = 14 \modulo 12 = 2
        \end{equation*}
    \end{soluzione}
    \begin{soluzione}
        \emph{Soluzione 2:}
        Aggiunto le 21 ore alle 5 attuali:

        \begin{equation*}
            21 + 5 = 26
        \end{equation*}

        Tolgo i giri inutili:

        \begin{equation*}
            26 \modulo 12 = 2
        \end{equation*}
    \end{soluzione}

Per come leggiamo l'orologio, le ore 2 e le ore 14 corrispondono alla stessa posizione della lancetta delle ore. Siamo talmente abituati che aggiungiamo in automatico 12 ore quando siamo nel pomeriggio.

Se estendiamo questo concetto, ogni volta che aggiungiamo 12 ore la lancetta delle ore ritorna esattamente nel punto di partenza. Quindi possiamo dire che le 2, le 14, le 26, le 38, \dots sono di fatto la stessa ora. Inoltre, visto che abbiamo a che fare con i numeri interi che comprendono anche i numeri negativi, possiamo aggiungere a questi anche le ore -10, -22, -34, \dots.

Cosa hanno in comune tutti questi numeri? Se ne calcoliamo la divisioni per 12, tutti questi numeri presentano lo stesso resto:

\begin{gather*}
    2 \modulo 12 = 2 \\
    14 \modulo 12 = 2 \\
    26 \modulo 12 = 2 \\
    \dots \\
    -10 \modulo 12 = 2 \\
    -22 \modulo 12 = 2 \\
    \dots
\end{gather*}

Possiamo quindi costruire 12 insiemi: uno per ogni ora del nostro quadrante:

\begin{gather*}
    \{\dots, -24, -12, 0, 12, 24, \dots\} \\
    \{\dots, -23, -11, 1, 13, 25, \dots\} \\
    \{\dots, -22, -10, 2, 14, 25, \dots\} \\
    \{\dots, -21, -9, 3, 15, 27, \dots\} \\
    \dots \\
    \{\dots, -13, -1, 11, 23, 35, \dots\} 
\end{gather*}

Nota però un dettaglio. Sul quadrante del nostro orologio abbiamo un solo numero per ciascuno di questi insiemi. I numeri presenti sul quadrante corrispondono al resto della divisione per 12 degli elementi di questi insiemi\dots tranne in un caso.

Se prendo gli elementi del primo insieme, $\{\dots, -24, -12, 0, 12, 24, \dots\}$, e ne faccio la divisione per 12, ottengono come resto il numero 0. Mentre sul quadrante del nostro orologio troviamo il numero 12. Per questo motivo i matematici preferirebbero avere il numero 0 al posto del numero 12.

Questi insiemi hanno molte caratteristiche interessanti. Per esempio se scelgo due insiemi e sommo un qualunque elemento del primo insieme con un qualunque elemento del secondo insieme ottengo sempre elementi di un terzo insieme: sempre lo stesso. Per esempio:

\begin{gather*}
    3 + 5 = 8 \\
    15 + 5 = 20 \\
    15 + 29 = 44 \\
\end{gather*}

Inoltre questo lavoro che abbiamo fatto con un orologio con 12 numeri può essere reso generico con orologi con una quantità qualunque di numeri.

\chapter{Aritmetica modulare}
\label{ch:aritmetica_modulare}

In questo capitolo $m$ indica un numero intero > 1.

\section{Congruenza modulo $m$}
\label{sec:congruenza_modulo_m}

\begin{definizione}[Congruenza modulo $m$]
    Due numeri interi $a$ e $b$ sono \textbf{congruenti modulo $m$}, e si scrive:
    \begin{equation*}
        \congruenti{a}{b}{m}
    \end{equation*}

    se, divisi per $m$, danno lo stesso resto:
    \begin{equation}
        \label{eq:def_congruenza} \congruenti{a}{b}{m} \sse a \modulo m = b \modulo m
    \end{equation}

\end{definizione}

Per esempio, tornando al quadrante dell'orologio, possiamo dire che 2 e 14 sono \emph{congruenti modulo 12}, in quanto:
\begin{equation*}
    \begin{cases}
        2 \modulo 12 = 2 \\
        14 \modulo 12 = 2
    \end{cases}
    \quad\Longrightarrow\quad
    \congruenti{2}{14}{12}
\end{equation*}

Se $m = 3$, allora i numeri 22 e 67 sono \emph{congruenti modulo 3}, quanto:
\begin{equation*}
    \begin{cases}
        22 \modulo 3 = 1 \\
        67 \modulo 3 = 1
    \end{cases}
    \quad\Longrightarrow\quad
    \congruenti{22}{67}{3}
\end{equation*}

\begin{teorema}
    \label{th:congruenza}
    \begin{equation}
        \label{eq:th_congruenza} \congruenti{a}{b}{m} \sse m \divisore (a - b)
    \end{equation}

    \mdfsubtitle{Dimostrazione $\Longrightarrow$}

    Faccio la divisione euclidea di $a$ e $m$ e ottengo un quoziente $q_a$ e un resto $r$:
    \begin{equation*}
        a = q_a m + r
    \end{equation*}

    Faccio la divisione euclidea di $b$ e $m$ e ottengo un quoziente $q_b$ e lo stesso resto $r$ in quanto $a$ e $b$ sono congruenti modulo $m$ per ipotesi:
    \begin{equation*}
        b = q_b m + r
    \end{equation*}

    Faccio ora la differenza tra $a$ e $b$ e ottengo:
    \begin{align*}
        a - b &= (q_a m + r) - (q_b m + r) = \\
        &= q_a m + r - q_b m - r = \\
        &= q_a m - q_b m = \\
        &= (q_a - q_b)m
    \end{align*}

    Quindi la divisione euclidea tra $a - b$ e $m$ dà resto 0, quindi $m$ divide la differenza $a - b$:
    \begin{equation*}
        m \divisore (a - b)
    \end{equation*}

\end{teorema}

Per esempio, nella matematica dell'orologio, 45 e 21 sono congruenti modulo 12 perché 12 divide $45 - 21$:
\begin{equation*}
    \congruenti{45}{21}{12} \sse 12 \divisore (45 - 21)
\end{equation*}

Dal momento che le formule~\eqref{eq:def_congruenza} e~\eqref{eq:th_congruenza} sono delle equivalenze logiche, potrei usare la~\eqref{eq:th_congruenza} come definizione di congruenza, e dimostrare la~\eqref{eq:def_congruenza} come teorema.

\begin{teorema}
    \label{th:divisore_della_differenza}
    \begin{equation*}
        m \divisore a \quad\land\quad m \divisore b \allora m \divisore (a - b)
    \end{equation*}

    \mdfsubtitle{Dimostrazione}

    Se $m$ divide $a$, significa che il resto della divisione $a : m$ è nullo.
    Lo stesso possiamo dire per $b$.
    Quindi dividendo $a$ e $b$ per $m$ otteniamo lo stesso resto, quindi i due numeri sono congruenti modulo $m$:
    \begin{equation*}
        m \divisore a \quad\land\quad m \divisore b \allora \congruenti{a}{b}{m}
    \end{equation*}

    Per il teorema~\ref{th:congruenza} possiamo inoltre affermare:
    \begin{equation*}
        m \divisore a \quad\land\quad m \divisore b \allora \congruenti{a}{b}{m} \allora m \divisore (a-b)
    \end{equation*}
\end{teorema}

\begin{teorema}
    La congruenza modulo $m$ è una relazione di equivalenza.

    \mdfsubtitle{Dimostrazione}

    Una relazione è di equivalenza se gode delle proprietà riflessiva, simmetrica e transitiva.

    La congruenza gode della proprietà \emph{riflessiva} infatti per la proprietà riflessiva dell'uguaglianza:
    \begin{equation*}
        \congruenti{a}{a}{m} \sse a \modulo m = a \modulo m
    \end{equation*}

    La congruenza gode della proprietà \emph{simmetrica} infatti:
    \begin{equation*}
        \congruenti{a}{b}{m} \allora a \modulo m = b \modulo m
    \end{equation*}
    quindi per la proprietà simmetrica dell'uguaglianza:
    \begin{equation*}
        \allora b \modulo m = a \modulo m \allora \congruenti{b}{a}{m}
    \end{equation*}

    La congruenza gode della proprietà \emph{transitiva} infatti:
    \begin{gather*}
        \congruenti{a}{b}{m} \allora a \modulo m = b \modulo m \\
        \congruenti{b}{c}{m} \allora b \modulo m = c \modulo m
    \end{gather*}
    quindi per la proprietà transitiva dell'uguaglianza:
    \begin{equation*}
        \allora a \modulo m = c \modulo m \allora \congruenti{a}{c}{m}
    \end{equation*}
\end{teorema}

\section{Le classi di congruenza}
\label{sec:classi_di_congruenza}

Come mostrato nel caso dell'orologio, la congruenza modulo $m$ divide l'insieme $\Z$ dei numeri interi in $m$ sottoinsiemi.
Per esempio, se consideriamo la congruenza modulo 3, abbiamo i seguenti insiemi:
\begin{gather*}
    \{\dots, -6, -3, 0, 3, 6, \dots\} \\
    \{\dots, -5, -2, 1, 4, 7, \dots\} \\
    \{\dots, -4, -1, 2, 5, 8, \dots\}
\end{gather*}

Questi insiemi hanno due caratteristiche significative:
\begin{itemize}
    \item ogni numero intero appartiene a uno di questi insiemi, ovvero l'unione degli insiemi forma l'insieme $\Z$;
    \item ogni numero intero appartiene a uno solo di questi insiemi, ovvero gli insiemi sono tra loro disgiunti.
\end{itemize}

Quando si presentano queste due caratteristiche si dice che gli insiemi formano una \textbf{partizione} dell'insieme di partenza, quindi possiamo dire che i tre insiemi sopra presentati \emph{partizionano} l'insieme degli interi $\Z$.

\begin{definizione}[Classi di congruenza]
    Gli insiemi che formano una partizione degli interi in base a una congruenza modulo $m$ si chiamano \textbf{classi di congruenza}.
\end{definizione}

Per indicare la classe di congruenza a cui appartiene un interno $a$, relativamente al modulo $m$, si utilizza la seguente notazione:
\begin{equation*}
    [a]_m
\end{equation*}

Talvolta, quando si lavora con un unico modulo o quando è chiaro quale modulo si sta usando, si può omettere il pedice $m$.

L'insieme $[a]_m$ contiene tutti i numeri interi che sono congruenti modulo $m$ dell'elemento $a$ scelto, per cui possiamo scrivere:
\begin{equation*}
    [a]_m = \{z \in \Z \mid \congruenti{z}{a}{m}\}
\end{equation*}

Tra tutti gli elementi di una classe di congruenza consideriamo rappresentante privilegiato quello positivo e minore di $m$.
Per cui i rappresentanti delle classi di congruenza a modulo 3 sono:
\begin{equation*}
    [0]_3, [1]_3, [2]_3
\end{equation*}

Quando vorremo riferirci alle classi di congruenza useremo normalmente questi rappresentanti privilegiati.

È immediato notare che i rappresentanti privilegiati coincidono con i resti della divisione per $m$ di tutti gli elementi di ciascuna classe di congruenza.
Per cui vale anche questa definizione:
\begin{equation*}
    [a]_m = \{z \in \Z \mid z \modulo m = a\} \quad \text{con } 0 \le a < m
\end{equation*}

Quindi possiamo dedurre le seguenti equivalenze:
\begin{gather*}
    [z]_m = [a]_m \sse z \modulo m = a \quad \text{con } 0 \le a < m \\
    [z]_m = [0]_m \sse z \modulo m = 0 \sse m \divisore z
\end{gather*}

Infine è immediato osservare che:
\begin{gather*}
    [m]_m = [0]_m \\
    [am]_m = [0]_m
\end{gather*}

Le classi di congruenza formano un nuovo insieme che indicheremo con $\Z_m$:
\begin{equation*}
    \Z_m= \{ [a]_m \mid 0 \le a < m \}
\end{equation*}

Per esempio:
\begin{gather*}
    \Z_2 = \{[0]_2, [1]_2\} \\
    \Z_3 = \{[0]_3, [1]_3, [2]_3\} \\
    \Z_4 = \{[0]_4, [1]_4, [2]_4, [3]_4\} \\
\end{gather*}

\section{Operazioni tra classi di congruenza}
\label{sec:operazioni_tra_classi_di_congruenza}

Quello che vogliamo scoprire, ora, è se è possibile inventare una nuova aritmetica che lavori con la classi di congruenza.\footnote{Non dovrebbe stupirti questo programma di costruire una nuova aritmetica basata sulle classi di congruenza. Dopotutto anche le frazioni sono della classi di congruenza, ricordi? In particolare:
\begin{equation*}
    \dfrac{a}{b} = \{ (x, y) \in \Z^2 \mid \exists p \in \Z, p \ne 0: x = pa; y = pb\}
\end{equation*}
Per ciascuna coppia ordinata il primo numero è un numeratore, mentre il secondo numero è un denominatore.

Per esempio:
\begin{equation*}
    \dfrac{2}{3} = \{ \dots, (-4, -6), (-2,-3), (2, 3), (4, 6), (6, 9), \dots \}
\end{equation*}
}

\begin{teorema}
    Se
    \begin{equation*}
        \congruenti{a_1}{a_2}{m} \quad\land\quad \congruenti{b_1}{b_2}{m}
    \end{equation*}
    allora:
    \begin{gather*}
        \congruenti{a_1 + b_1}{a_2 + b_2}{m} \\
        \congruenti{a_1 \cdot b_1}{a_2 \cdot b_2}{m} \\
        \congruenti{a_1 - b_1}{a_2 - b_2}{m}
    \end{gather*}

    \mdfsubtitle{Dimostrazione della somma}

    Se $a_1$ e $a_2$ sono congruenti, allora le divisioni hanno lo stesso resto, che chiamo $r_a$;
    similmente chiamo $r_b$ il resto delle divisioni di $b_1$ e $b_2$
    \begin{gather*}
        a_1 = q_{a_1} m + r_a \\
        a_2 = q_{a_2} m + r_a \\
        b_1 = q_{b_1} m + r_b \\
        b_2 = q_{b_2} m + r_b
    \end{gather*}

    Calcoliamo ora le somme che ci interessano:
    \begin{gather*}
        a_1 + b_1 = q_{a_1} m + r_a + q_{b_1} m + r_b = (q_{a_1} + q_{b_1}) m + (r_a + r_b) \\
        a_2 + b_2 = q_{a_2} m + r_a + q_{b_2} m + r_b = (q_{a_2} + q_{b_2}) m + (r_a + r_b)
    \end{gather*}
    Risulta quindi che le divisioni di $a_1 + b_1$ e di $a_2 + b_2$ per $m$ hanno entrambi il resto uguale a $r_a + r_b$.
    Quindi le due somme sono congruenti.
\end{teorema}

Questo teorema ha una conseguenza importantissima.
Prendiamo due classi di congruenza: $[a]_m$ e $[b]_m$.
Qualunque sia l'elemento che prendo della prima classe e qualunque sia l'elemento che prendo dalla seconda classe, la loro somma apparterrà sempre alla medesima classe.

Per esempio, qualsiasi elemento prenda da $[1]_4$ e qualsiasi elemento prenda da $[2]_4$, la loro somma apparterrà sempre alla classe $[3]_4$.

Questo mi permette d'introdurre una nuova operazione.

\begin{definizione}[Somma tra classi]
    Definisco la somma tra due classi di equivalenza come la classe di equivalenza che contiene la somma di due loro rappresentanti:
    \begin{equation}
        \label{eq:somma} [a]_m + [b]_m = [a + b]_m
    \end{equation}
\end{definizione}

Similmente:

\begin{definizione}[Prodotto tra classi]
    Definisco il prodotto tra due classi di equivalenza come la classe di equivalenza che contiene il prodotto di due loro rappresentanti:
    \begin{equation}
        \label{eq:prodotto} [a]_m \cdot [b]_m = [a \cdot b]_m
    \end{equation}
\end{definizione}

\begin{definizione}[Differenza tra classi]
    Definisco la differenza tra due classi di equivalenza come la classe di equivalenza che contiene la differenza di due loro rappresentanti:
    \begin{equation}
        \label{eq:differenza} [a]_m - [b]_m = [a - b]_m
    \end{equation}
\end{definizione}

Queste nuove operazioni mi permettono di semplificare i calcoli quando voglio calcolare i resti delle divisioni.

\begin{esercizio}
    Calcola il seguente resto:
    \begin{equation*}
        (73 \cdot 26 + 94) \modulo 7
    \end{equation*}

    \mdfsubtitle{Soluzione}
    Calcolare il resto richiesto corrisponde a determinare qual è la classe di congruenza del risultato dell'operazione al dividendo:
    \begin{align*}
        [73 \cdot 26 + 94]_7 &= [73 \cdot 26]_7 + [94]_7 &&\text{per la~\eqref{eq:somma}} \\
        &= [73]_7 \cdot [26]_7 + [94]_7 &&\text{per la~\eqref{eq:prodotto}} \\
        &= [3]_7 \cdot [5]_7 + [3]_7 &&\text{per congruenza modulo 7} \\
        &= [3 \cdot 5]_7 + [3]_7 &&\text{per la~\eqref{eq:prodotto}} \\
        &= [15]_7 + [3]_7 \\
        &= [15 + 3]_7 &&\text{per la~\eqref{eq:somma}} \\
        &= [18]_7 \\
        &= [4]_7 &&\text{per congruenza modulo 7}
    \end{align*}

    Quindi:
    \begin{equation*}
        (73 \cdot 26 + 94) \modulo 7 = 4
    \end{equation*}
\end{esercizio}

\begin{esercizio}
    Determinare per quali valori interi di $n$ la seguenti espressione è intera.
    \[
        \dfrac{n+5}{n+2}
    \]

    \mdfsubtitle{Soluzione utilizzando le classi di congruenza}

    Affinché l'espressione data sia intera è necessario che $n+2$ divida $n+5$.
    Pertanto, utilizzando le classi di congruenza modulo $n+2$ deve risultare:
    \[
        [n+5]_{n+2} = [0]_{n+2}
    \]

    Posso semplificare questa espressione scrivendo:
    \begin{gather*}
        [n+2+3]_{n+2} = [0]_{n+2} \\
        [n+2]_{n+2}+[3]_{n+2} = [0]_{n+2} \\
        [0]_{n+2}+[3]_{n+2} = [0]_{n+2} \\
        [3]_{n+2} = [0]_{n+2}
    \end{gather*}

    Quindi dobbiamo verificare quando $n+2$ divide 3.
    Il numero 3 ha solo due divisori positivi: l'1 e il 3.
    Ma devo considerare anche i divisori negativi.
    Quindi dobbiamo considerare quattro casi:
    \begin{align*}
        n+2 = 3 &\allora n = 1 \\
        n+2 = 1 &\allora n = -1 \\
        n+2 = -1 &\allora n = -3 \\
        n+2 = -3 &\allora n = -5
    \end{align*}

    \mdfsubtitle{Soluzione utilizzando le frazioni}

    Scomponiamo la frazione nel seguente modo:
    \begin{equation*}
        \dfrac{n+5}{n+2} = \dfrac{n+2 + 3}{n+2} = \dfrac{n+2}{n+2} + \dfrac{3}{n+2} = 1 + \dfrac{3}{n+2}
    \end{equation*}

    Quindi l'espressione è un intero se $n+2$ divide 3.

    Poi si procede come per la soluzione precedente.

    \mdfsubtitle{Risposta}

    I valori di $n$ cercati sono $-5$, $-3$, $-1$ e $1$.
    Questi corrispondono ai seguenti valori dell'espressione.
    \begin{align*}
        n = -5 &\allora \dfrac{n+5}{n+2} = \dfrac{-5+5}{-5+2} = \dfrac{0}{-3} = 0 \\
        n = -3 &\allora \dfrac{n+5}{n+2} = \dfrac{-3+5}{-3+2} = \dfrac{2}{-1} = -2 \\
        n = -1 &\allora \dfrac{n+5}{n+2} = \dfrac{-1+5}{-1+2} = \dfrac{4}{1} = 4 \\
        n = 1 &\allora \dfrac{n+5}{n+2} = \dfrac{1+5}{1+2} = \dfrac{6}{3} = 2 \\
    \end{align*}

\end{esercizio}

\section{Proprietà delle operazioni tra classi}
\label{sec:proprieta_operazioni_classi}

\begin{teorema}[Proprietà della somma]
    \begin{itemize}
        \item proprietà commutativa \\ $[a]_m + [b]_m = [b]_m + [a]_m$
        \item proprietà associativa \\ $([a]_m + [b]_m) + [c]_m = [a]_m + ([b]_m + [c]_m)$
        \item $[0]_m$ è l'elemento neutro della somma \\ $[0]_m + [a]_m = [a]_m + [0]_m = [0]_m$
        \item ogni $[a]_m$ ha un \textbf{opposto}: $[m-a]_m$ \\ $[a]_m + [m-a]_m = [m-a]_m + [a]_m = [0]_m$
    \end{itemize}
\end{teorema}

\begin{teorema}[Proprietà del prodotto]
    \begin{itemize}
        \item proprietà commutativa \\ $[a]_m \cdot [b]_m = [b]_m \cdot [a]_m$
        \item proprietà associativa \\ $([a]_m \cdot [b]_m) \cdot [c]_m = [a]_m \cdot ([b]_m \cdot [c]_m)$
        \item $[1]_m$ è l'elemento neutro del prodotto \\ $[0]_m \cdot [a]_m = [a]_m \cdot [0]_m = [0]_m$
    \end{itemize}
\end{teorema}

Sebbene il prodotto si \emph{comporti bene} in $\Z_m$, perdiamo alcune proprietà che invece
abbiamo nell'aritmetica in $\Z$.

Un esempio è la \textbf{legge dell'annullamento del prodotto}: in $\Z$ se il prodotto di due numeri è uguale a zero,
almeno uno dei due numeri è nullo:
\begin{equation*}
    ab = 0 \sse a = 0 \lor b = 0
\end{equation*}

In $\Z_m$ questo non è generalmente vero, per esempio:
\begin{equation*}
    [2]_{10} \cdot [5]_{10} = [10]_{10} = [0]_{10}
\end{equation*}


\chapter{Criteri di congruenza e criteri di divisibilità}

In questo capitolo avremo spesso a che fare con le cifre di cui è composto un numero. E avremo bisogno di sostituire le singole cifre con delle lettere. Mettiamo una riga sopra ad una rappresentazione di un numero per indicare la giustapposizione delle cifre e non il prodotto dei numeri. Per esempio:
\begin{equation*}
    a=1, b=2, c=3 \allora \overline{abc} = 123.
\end{equation*}

Dobbiamo tenere presente, inoltre, la scomposizione dei numeri in unità, decine, centinaia, eccetera. Quindi:
\begin{equation*}
    \overline{abc} = a \cdot 100 + b \cdot 10 + c
\end{equation*}

\section{Criteri di congruenza}

I criteri di congruenza servono per determinare a che cosa è congruente un intero fissato.

\begin{mdframed}
    \begin{teorema}
        Un intero è congruente modulo 2 alla sua cifra delle unità.
    \end{teorema}
    \begin{proof}
        Presento la dimostrazione per numeri di 3 cifre. La dimostrazione è facilmente estendibile ad un qualunque numero di cifre.

        Consideriamo il numero $\overline{abc}$. Risulta:

        \begin{align*}
            [\overline{abc}]_2 &= [a \cdot 100 + b \cdot 10 + c]_2 = \\
            &= [a]_2 \cdot [100]_2 + [b]_2 \cdot [10]_2 + [c]_2 = \\
            &= [a]_2 \cdot [0]_2 + [b]_2 \cdot [0]_2 + [c]_2 = \\
            &= [0]_2 + [0]_2 + [c]_2 = \\
            &= [c]_2
        \end{align*}
    \end{proof}
\end{mdframed}

\begin{mdframed}
    \begin{teorema}
        Un intero è congruente modulo 3 alla somma delle sue cifre.
    \end{teorema}
    \begin{proof}
        Presento la dimostrazione per numeri di 3 cifre. La dimostrazione è facilmente estendibile ad un qualunque numero di cifre.

        Consideriamo il numero $\overline{abc}$. Risulta:

        \begin{align*}
            [\overline{abc}]_3 &= [a \cdot 100 + b \cdot 10 + c]_3 = \\
            &= [a]_3 \cdot [100]_3 + [b]_2 \cdot [10]_3 + [c]_3 = \\
            &= [a]_3 \cdot [1]_3 + [b]_3 \cdot [1]_3 + [c]_3 = \\
            &= [a]_3 + [b]_3 + [c]_3 = \\
            &= [a + b + c]_3
        \end{align*}
    \end{proof}
\end{mdframed}

Senza dare le dimostrazioni (lasciate alla buona volontà del lettore) aggiungo gli altri criteri di congruenza nella tabella~\ref{tab:criteri_congruenza}.

\begin{table}[tp]
    \begin{mdframed}    
        \label{tab:criteri_congruenza}
        \centering
        \begin{tabular}{c|rcl}
            \toprule
            Modulo & Criterio & di & congruenza \\
            \midrule
            modulo 2 & $[\overline{abcde}]_2 $ & $ = $ & $ [e]_2$ \\
            modulo 3 & $[\overline{abcde}]_3 $ & $ = $ & $ [a + b + c + d + e]_3$ \\
            modulo 4 & $[\overline{abcde}]_4 $ & $ = $ & $ [\overline{de}]_4$ \\
            modulo 5 & $[\overline{abcde}]_5 $ & $ = $ & $ [e]_5$ \\
            modulo 8 & $[\overline{abcde}]_8 $ & $ = $ & $ [\overline{cde}]_8$ \\
            modulo 9 & $[\overline{abcde}]_9 $ & $ = $ & $ [a + b + c + d + e]_9$ \\
            modulo 10 & $[\overline{abcde}]_{10} $ & $ = $ & $ [e]_{10}$ \\
            modulo 11 & $[\overline{abcde}]_{11} $ & $ = $ & $ [a - b + c - d + e]_{11}$ \\
            \bottomrule
        \end{tabular}
        \caption{Criteri di congruenza}
    \end{mdframed}
\end{table}

\section{Criteri di divisibilità}

Dai criteri di congruenza derivano i criteri di divisibilità.

\begin{mdframed}
    \begin{teorema}
        Un intero è divisibile per 2 se e solo se è pari la cifra delle unità.
    \end{teorema}
    \begin{proof}
        Per il criterio di congruenza modulo 2:
        \begin{equation*}
            [\overline{abcde}]_2 = [e]_2
        \end{equation*}
        Affinché il numero $\overline{abcde}$ sia divisibile per 2, deve essere congruente a 0, ovvero il numero deve appartenere alla classe di congruenza dello 0:
        \begin{equation*}
            [\overline{abcde}]_2 = [e]_2 = [0]_2
        \end{equation*}
        Da questo deduciamo che un numero è divisibile per 2 se la sua cifra delle unità è divisibile per 2, ovvero è pari.
    \end{proof}
\end{mdframed}

\begin{mdframed}
    \begin{teorema}
        Un intero è divisibile per 3 se e solo se la somma delle sue cifre è divisibile per 3.
    \end{teorema}
    \begin{proof}
        Per il criterio di congruenza modulo 3:
        \begin{equation*}
            [\overline{abcde}]_3 = [a + b + c + d + e]_3
        \end{equation*}
        Affinché il numero $\overline{abcde}$ sia divisibile per 3, deve essere congruente a 0, ovvero il numero deve appartenere alla classe di congruenza dello 0:
        \begin{equation*}
            [\overline{abcde}]_3 = [a + b + c + d + e]_3 = [0]_2
        \end{equation*}
        Da questo deduciamo che un numero è divisibile per 3 se la somma delle sue cifre è divisibile per 3.
    \end{proof}
\end{mdframed}

Le dimostrazioni dei rimanenti criteri di visibilità sono identiche a queste due presentate, e conducono ai criteri di divisibità che già conoscete.

Aggiungiamo un criterio di divisibilità che non derivano da un criterio di congruenza:

\begin{mdframed}
    \begin{teorema}
        Un intero è divisibile per 7 se e solo se è divisibile per 7 l'intero costituito nel seguente modo: si prende l'intero iniziale privato della cifra delle unità e gli si sottrae il doppio della cifra delle unità.
    \end{teorema}
    \begin{proof}[Dimostrazione parte se]
        Per semplicità presentiamo la dimostrazione per numeri di due cifre. La dimostrazione è facilmente estendibile a numeri di qualunque lunghezza.

        Quello che vogliamo dimostrare è che il numero $\overline{ab}$ è divisibile per 7 se è divisibile il numero costruito prendendo l'intero iniziale privato della cifra delle unità (e quindi rimane solo la cifra delle decine) a cui si sottrae il doppio della cifra delle unità, ovvero se è divisibile per 7 il numero $a - 2b$. Quindi, passando alle classi di congruenza, dobbiamo dimostrare che:
        \begin{equation*}
            [a - 2b]_7 = [0]_7 \allora [\overline{ab}]_7 = [0]_7
        \end{equation*}
        Dall'ipotesi con cui stiamo lavorando possiamo ricavare che:
        \begin{equation*}
            [a]_7 = [2b]_7
        \end{equation*}
        Calcoliamo allora la classe di congruenza del numero $\overline{ab}$:
        \begin{align*}
            [\overline{ab}]_7 &= [a \cdot 10 + b]_7 = \\
            &= [a]_7 \cdot [10]_7 + [b]_7 =\\
            &= [2b]_7 \cdot [3]_7 + [b]_7 =\\
            &= [6b]_7 + [b]_7 =\\
            &= [7b]_7 =\\
            &= [0]_7
        \end{align*}
        
    \end{proof}
\end{mdframed}




\chapter{Massimo Comun Divisore}
\label{ch:mcd}

Introduciamo una notazione alternativa per il MCD di due numeri (si usa solo con due numeri, non con tre o più):
\begin{equation*}
    (a, b) = MCD(a, b)
\end{equation*}

\section{Metodo delle sottrazioni successive}
\label{sec:metodo_sottrazioni_successive}

Il teorema~\ref{th:divisore_della_differenza} ci permette d'introdurre una tecnica di calcolo del MCD alternativa a quella tipicamente insegnata a scuola (scomposizione e ricerca dei fattori comuni).

Dati due numeri $a$ e $b$, l'MCD è il massimo dei numeri che contiene divisori di entrambi i numeri.
Ma il teorema ci dice che se i due numeri hanno uno stesso divisore $m$, tale numero divide anche la differenza $a - b$.

\begin{definizione}[Metodo delle sottrazioni successive]

Possiamo calcolare l'MCD di due numeri sostituendo il maggiore con la differenza dei due.
E poi reiteriamo questa sostituzione fino a quando i due numeri residui sono uguali.

\end{definizione}

\begin{esercizio}
    Calcolare il MCD tra 6120 e 3780.

    \mdfsubtitle{Soluzione}
    \begin{align*}
        (6120, 3780) &= (2340, 3780) &&\text{perché } 6120 - 3780 = 2340 \\
        &= (2340, 1440) &&\text{perché } 3780 - 2340 = 1440 \\
        &= (900, 1440) &&\text{perché } 2340 - 1440 = 900 \\
        &= (900, 540) &&\text{perché } 1440 - 900 = 540 \\
        &= (360, 540) &&\text{perché } 900 - 540 = 360 \\
        &= (360, 180) &&\text{perché } 540 - 360 = 180 \\
        &= (180, 180) &&\text{perché } 360 - 180 = 180 \\
        &= 180
    \end{align*}
\end{esercizio}

Se poi dobbiamo calcolare anche il mcm, ci basta ricordare la relazione fondamentale:
\begin{equation*}
    a \cdot b = MCD(a, b) \cdot mcm(a, b)
\end{equation*}

\section{Esercizi}
\label{sec:esercizi_mcd}

\textbf{Esercizio 1}
Dire se la seguente affermazione è vera o falsa;
$p$ denota un numero primo, $a$ e $b$ generici numeri interi.
\begin{equation*}
    p \divisore (a, b) \allora p \divisore (a^2, ab)
\end{equation*}
\chapter{Esercizi e quiz}
\label{ch:quiz}

\section{Dall'eserciziario dello stage senior}
\label{sec:quiz_stage_senior}

\begin{esercizio}
    \label{ex:stage_senior_41}
    Determinare per quali valori interi di $n$ le seguenti espressioni sono intere.

    \begin{equation*}
        \dfrac{n + 3}{n + 1} \quad \dfrac{3n + 10}{n + 2} \quad \dfrac{n + 7}{2n + 1} \quad \dfrac{3a + 1}{2a + 3} \quad
        \dfrac{15 - 3n}{2n^2 + 1}
    \end{equation*}
\end{esercizio}

\section{Quiz dalle gare distrettuali}
\label{sec:quiz_gare_distrettuali}

\begin{esercizio}[Gare distrettuali 2017]
    \label{ex:distrettuali_2017_13}
    Il ricco Creso compra 88 vasi identici.
    Il prezzo di ognuno di essi, espresso in dracme, è un numero intero (lo stesso per tutti gli 88 vasi).
    Sappiamo che Creso paga un totale di $a1211b$ dracme, dove $a$, $b$ sono cifre da determinare
    (e che possono essere distinte o meno).
    Quante dracme costa un singolo vaso?
\end{esercizio}

\begin{esercizio}[Gare distrettuali 2018]
    \label{ex:distrettuali_2018_1}
    Una gara di matematica consta di 90 domande a risposta multipla.
    Camilla ha risposto a tutte le domande: quale dei seguenti non può essere il punteggio totalizzato da Camilla,
    sapendo che una risposta corretta vale 5 punti e una risposta sbagliata vale $-1$ punto?

    (A) $-78$ \quad (B) 116 \quad (C) 204 \quad (D) 318 \quad (E) 402
\end{esercizio}

\begin{esercizio}[Gare distrettuali 2018]
    \label{ex:distrettuali_2018_15}
    \begin{enumerate}
        \item Trovare tutti gli interi positivi $n$ di due cifre che godano della seguente proprietà: entrambi
        gli interi che si ottengono cancellando una delle due cifre della rappresentazione decimale
        di $n$ sono divisori (interi positivi) di $n$.
        \item Sia $n > 10$ un intero che si scrive con $k$ cifre decimali, tutte diverse da zero.
        Supponiamo che ciascuno degli interi ottenuti cancellando una delle $k$ cifre della rappresentazione decimale
        di $n$ sia un divisore (intero positivo) di $n$.
        Mostrare che necessariamente $k = 2$. \\
        Esempio.
        Per $n = 123$ si ha $k = 3$, e gli interi ottenuti cancellando cifre di $n$ sono 23, 13 e 12.
    \end{enumerate}
\end{esercizio}

\begin{esercizio}[Gare distrettuali 2019]
    \label{ex:distrettuali_2019}
    Jacopo ha a disposizione 6 colori (tra cui il bianco) per colorare tutti i numeri interi.
    Vuole rispettare però queste condizioni: $n$ e $n + 5$ devono avere lo stesso colore per ogni $n$ intero e
    inoltre se $ab$ è bianco, allora almeno uno tra $a$ e $b$ deve essere bianco.
    In quanti modi Jacopo può colorare gli interi?

    (A) 156 \quad (B) 656 \quad (C) 3181 \quad (D) 3906 \quad (E) 3936
\end{esercizio}

\begin{esercizio}[Gare distrettuali 2021]
    \label{ex:distrettuali_2021}
    Quanti sono i numeri di 6 cifre divisibili per 33 che siano palindromi, cioè che rimangano uguali
    se letti da destra verso sinistra?

    (A) 30 \quad (B) 33 \quad (C) 300 \quad (D) 333 \quad (E) Nessuna delle precedenti.
\end{esercizio}

\begin{esercizio}[Gare distrettuali 2023, es. 9]
    \label{ex:distrettuali_2023_9}
    La successione $a_n$ è costruita nel modo seguente:
    $a_1$, $a_2$ sono interi compresi fra 1 e 9 (estremi inclusi);
    per $n \ge 3$, se la somma fra $a_{n-1}$ e $a_{n-2}$ consta di una sola cifra, allora tale somma è il valore di $a_n$;
    se invece $a_{n-1} + a_{n-2}$ ha più di una cifra, la somma delle sue cifre sarà il valore di $a_n$
    (ad esempio, se $a_4 = 7$ e $a_5 = 8$, allora $a_6 = 6$ in quanto $7 + 8 = 15$ e $1 + 5 = 6$).
    Quante sono le scelte possibili della coppia $(a_1, a_2)$ tali che si abbia $a_{2023} = 9$?

    (A) 1 \quad (B) 3 \quad (C) 9 \quad (D) 27 \quad (E) 81
\end{esercizio}

\section{Altri quiz}
\label{sec:quiz_altri}

\begin{esercizio}[Proposto da Francesco P.]
    \label{ex:francesco_1}
    Quanti sono i numeri nella forma $\overline{abcabc}$ divisibili per 2023?
\end{esercizio}

\begin{esercizio}[Proposto da Francesco P.]
    \label{ex:francesco_2}
    Quanti sono i numeri nella forma $\overline{abcabc}$ che sono quadrati perfetti?
\end{esercizio}
\chapter{Soluzioni}
\label{ch:soluzioni}


\begin{soluzione}
    \mdfsubtitle{Soluzione esercizio~\ref{ex:stage_senior_41}}

    Affinché $\dfrac{n + 3}{n + 1}$ sia intero è necessario che $n+1$ divida $n+3$:
    \begin{gather*}
        [n + 3]_{n + 1} = [0]_{n+1} \\
        [n + 1 + 2]_{n + 1} = [0]_{n+1} \\
        [n + 1]_{n+1} + [2]_{n + 1} = [0]_{n+1} \\
        [2]_{n + 1} = [0]_{n+1} \\
    \end{gather*}

    Quindi $n+1$ deve dividere 2.
    I divisori (positivi e negativi) di 2 sono $-2$, $-1$, 1 e 2:
    \begin{align*}
        n + 1 = -2 &\allora n = -3 \\
        n + 1 = -1 &\allora n = -2 \\
        n + 1 = 1 &\allora n = 0 \\
        n + 1 = 2 &\allora n = 1
    \end{align*}

    Affinché $\dfrac{3n + 10}{n + 2}$ sia intero è necessario che $n+2$ divida $3n+10$:
    \begin{gather*}
        [3n+10]_{n+2} = [0]_{n+2} \\
        [3n + 6 + 4]_{n+2} = [0]_{n+2} \\
        [3n + 6]_{n+2} + [4]_{n+2} = [0]_{n+2} \\
        [4]_{n+2} = [0]_{n+2}
    \end{gather*}

    Quindi $n+2$ deve dividere 4.
    I divisori (positivi e negativi) di 4 sono $-4$, $-2$, $-1$, 1, 2 e 4:
    \begin{align*}
        n+2 = -4 &\allora n = -6 \\
        n+2 = -2 &\allora n = -4 \\
        n+2 = -1 &\allora n = -3 \\
        n+2 = 1 &\allora n = -1 \\
        n+2 = 2 &\allora n = 0 \\
        n+2 = 4 &\allora n = 2
    \end{align*}

    Affinché $\dfrac{n + 7}{2n + 1}$ sia intero è necessario che $2n+1$ divida $n+7$.
    Però in questo caso non riusciamo ad isolare a numeratore un multiplo del denominatore, a causa di quel fattore 2
    che moltiplica $n$ a denominatore.

    Osserviamo che il denominatore è un numero dispari quindi se moltiplichiamo il numeratore per 2 non rischiamo
    di aggiungere fattori che sono presenti nel denominatore.
    Quindi $2n+1$ divide $n+7$ se e solo se $2n+1$ divide $2(n+7)$:
    \begin{gather*}
        [2n +14]_{2n+1} = [0]_{2n + 1} \\
        [13]_{2n + 1} = [0]_{2n + 1}
    \end{gather*}

    I divisori di 13 sono $-13$, $-1$, 1 e 13:
    \begin{align*}
        2n + 1 = -13 &\allora n = -7 \\
        2n + 1 = -1 &\allora n = -1 \\
        2n + 1 = 1 &\allora n = 0 \\
        2n + 1 = 13 &\allora n = 6
    \end{align*}

    Affinché $\dfrac{3a + 1}{2a + 3}$ sia intero è necessario che $2a+3$ divida $3a+1$.
    Il denominatore è dispari, quindi posso moltiplicare il numeratore per 2:
    \begin{gather*}
        [6a + 2]_{2a + 3} = [0]_{2a + 3} \\
        [6a + 9 - 7]_{2a + 3} = [0]_{2a+3} \\
        [-7]_{2a + 3} = [0]_{2a+3}
    \end{gather*}

    Quindi $2a + 3$ deve dividere 7:
    \begin{align*}
        2a + 3 = -7 &\allora a = -5 \\
        2a + 3 = -1 &\allora a = -2 \\
        2a + 3 = 1 &\allora a = -1 \\
        2a + 3 = 7 &\allora a = 2
    \end{align*}

    Per l'ultimo caso dobbiamo ricorrere a qualche altro stratagemma, perché il denominatore scritto in quel modo
    non mi permette di operare semplificazioni sul numeratore.

    Possiamo allora tentare di restringere l'analisi tenendo conto del fatto che il divisore deve essere (in valore
    assoluto) minore o uguale al denominatore (in valore assoluto).
    Nella frazione considerata, inoltre, abbiamo il vantaggio che il denominatore è sicuramente positivo, quindi
    possiamo semplificare la condizione in:
    \begin{gather*}
        2n^2 + 1 \le 15 - 3n \\
        2n^2 + 3n - 14 \le 0 \\
        -\dfrac{7}{2} \le n \le 2
    \end{gather*}

    Quindi gli unici casi da analizzare sono i seguenti:
    \begin{align*}
        n = -3 &\allora \dfrac{15 - 3n}{2n^2 + 1} = \dfrac{24}{19} &\text{non accettabile} \\
        n = -2 &\allora \dfrac{15 - 3n}{2n^2 + 1} = \dfrac{21}{9} &\text{non accettabile} \\
        n = -1 &\allora \dfrac{15 - 3n}{2n^2 + 1} = \dfrac{18}{3} &\text{accettabile} \\
        n = 0 &\allora \dfrac{15 - 3n}{2n^2 + 1} = \dfrac{15}{1} &\text{accettabile} \\
        n = 1 &\allora \dfrac{15 - 3n}{2n^2 + 1} = \dfrac{12}{3} &\text{accettabile} \\
        n = 2 &\allora \dfrac{15 - 3n}{2n^2 + 1} = \dfrac{9}{9} &\text{accettabile}
    \end{align*}
\end{soluzione}

\begin{soluzione}
    \mdfsubtitle{Soluzione esercizio~\ref{ex:distrettuali_2018_1}}
    Chiamiamo $n$ il numero di risposte corrette date da Camilla.
    Quindi ha sbagliato $90 - n$ risposte.

    Per le risposte corrette ottiene $5n$ punti.

    Per le risposte sbagliate ottiene $-1(90-n)$ punti.

    In totale ottiene:
    \[
        5n - 1(90 - n) = 5n - 90 + n = 6n - 90 = 6(n - 15)
    \]

    Quindi il punteggio di Camilla deve essere divisibile per 6.
    L'unico punteggio tra quelli proposti non divisibile per 6 è il 116.



\end{soluzione}

\begin{soluzione}
    \mdfsubtitle{Soluzione esercizio~\ref{ex:distrettuali_2018_15}: parte 1)}
    L'esercizio ci chiede di trovare tutti i numeri interi positivi di due cifre $\overline{ab}$ che sono divisibili
    per $a$ e per $b$:
    \begin{gather*}
        a \divisore \overline{ab} \\
        b \divisore \overline{ab}
    \end{gather*}

    Il divisore non può essere nullo, quindi $a$ e $b$ non sono nulli.

    Passando alle classi di congruenza, per quanto riguarda il divisore $a$ abbiamo che:
    \begin{gather*}
        a \divisore \overline{ab} \\
        [\overline{ab}]_a = [0]_a \\
        [10]_a[a]_a + [b]_a = [0]_a \\
        [10]_a[0]_a + [b]_a = [0]_a \\
        [b]_a = [0]_a \\
        a \divisore b \\
        \exists k : b = ka
    \end{gather*}

    Invece per quanto riguarda il divisore $b$ abbiamo che:
    \begin{gather*}
        b \divisore \overline{ab} \\
        [\overline{ab}]_b = [0]_b \\
        [10]_b[a]_b + [b]_b = [0]_b \\
        [10]_b[a]_b + [0]_b = [0]_b \\
        [10]_b[a]_b = [0]_b \\
        [10a]_b = [0]_b \\
        b \divisore 10a \\
        ka \divisore 10a \\
        k \divisore 10
    \end{gather*}

    Riassumendo abbiamo ottenuto due condizioni che dobbiamo rispettare:
    \begin{gather*}
        b = ka \\
        k \divisore 10
    \end{gather*}

    I valori possibili di $k$ sono i divisori di 10 che fanno in modo che $b$ sia una cifra.
    Essi sono: 1, 2 e 5.

    Enumeriamo quindi i casi possibili:
    \begin{itemize}
        \item $k = 1 \,\Longrightarrow\, b = a$: ottengo i numeri 11, 22, 33, 44, 55, 66, 77, 88, 99;
        \item $k = 2 \,\Longrightarrow\, b = 2a$: ottengo i numeri 12, 24, 36, 48;
        \item $k = 5 \,\Longrightarrow\, b = 5a$: ottengo il numero 15.
    \end{itemize}

    \bigskip
    \mdfsubtitle{Risposta}
    Gli interi positivi cercati sono: 11, 12, 15, 22, 24, 33, 36, 44, 48, 55, 66, 77, 88, 99.

    \bigskip
    \mdfsubtitle{Soluzione esercizio~\ref{ex:distrettuali_2018_15}: parte 2)}
    Ora abbiamo a che fare con numeri di due o più cifre.
    Non possiamo più usare la notazione posizionale, ma possiamo comunque scomporre il numero $n$ in $10a + b$,
    dove $a$ è un qualunque numero intero maggiore di 1, e $b$ è una singola cifra.

    Il numero $n = 10a + b$ deve essere divisibile per $a$, quindi:
    \begin{gather*}
        a \divisore 10a + b \\
        [10a + b]_a = [0]_a \\
        [10]_a[a]_a + [b]_a = [0]_a \\
        [10]_a[0]_a + [b]_a = [0]_a \\
        [b]_a = [0]_a \\
        a \divisore b
    \end{gather*}

    Affinché $a$ sia un divisore di $b$ è necessario che $a \le b$.
    Visto che $b$ è una singola cifra, ovvero $b \le 9$, deduciamo che anche $a \le 9$.
    Quindi il numero numero $n = 10a + b$ è composto al più di 2 cifre.
\end{soluzione}

\begin{soluzione}
    \mdfsubtitle{Soluzione esercizio~\ref{ex:distrettuali_2019}}
    La condizione
    \begin{quotation}
        $n$ e $n + 5$ devono avere lo stesso colore per ogni $n$ intero
    \end{quotation}
    ci permette di operare considerando solo le classi di congruenza modulo 5.
    Infatti $n$ e $n + 5$ si trovano nella stessa classe:
    \begin{align*}
        [n + 5]_5 &= [n]_5 + [5]_5 = && \text{per somma di classi} \\
        &= [n]_5 + [0]_5 = && \text{perché $5 \modulo 5 = 0$} \\
        &= [n]_5 && \text{perché $[0]_5$ è l'elemento neutro della somma}
    \end{align*}

    Dal momento che in questo esercizio avremo a che fare solo con classi modulo 5 non verrà più specificato il modulo
    nelle notazioni.

    Gli elementi di ciascuna classe di congruenza avranno tutti lo stesso colore.
    Dobbiamo quindi determinare in quanti modi possiamo colorare le 5 classi.
    Nella colorazione dobbiamo rispettare anche la seconda condizione:
    \begin{quotation}
        se $ab$ è bianco, allora almeno uno tra $a$ e $b$ deve essere bianco.
    \end{quotation}

    Le classi di congruenza si comportano bene con il prodotto tra numeri, quindi posso modificare la condizione in:
    \begin{quotation}
        se $[ab]$ è bianca, allora almeno una tra $[a]$ e $[b]$ deve essere bianca.
    \end{quotation}

    Abbiamo quindi bisogno di capire come si comportano le moltiplicazioni con le classi di congruenza modulo 5.
    Costruiamo quindi la tavola pitagorica (o \emph{tavola di Cayley}) del prodotto tra le classi di congruenza:

    \begin{table}[H]
        \label{tab:distrettuali_2019}
        \centering
        \begin{tabular}{c|ccccc}
            $\cdot$ & $[0]$ & $[1]$ & $[2]$ & $[3]$ & $[4]$ \\
            \midrule
            $[0]$ & $[0]$ & $[0]$ & $[0]$ & $[0]$ & $[0]$ \\
            $[1]$ & $[0]$ & $[1]$ & $[2]$ & $[3]$ & $[4]$ \\
            $[2]$ & $[0]$ & $[2]$ & $[4]$ & $[1]$ & $[3]$ \\
            $[3]$ & $[0]$ & $[3]$ & $[1]$ & $[4]$ & $[2]$ \\
            $[4]$ & $[0]$ & $[4]$ & $[3]$ & $[2]$ & $[1]$
        \end{tabular}
    \end{table}

    Questa tabella ci permette di capire con quali prodotti otteniamo le classi di congruenza:
    \begin{itemize}
        \item $[0] = [0][a] \,\,\forall [a]$;
        \item $[1] = [1][1] = [2][3] = [4][4]$;
        \item $[2] = [1][2] = [3][4]$;
        \item $[3] = [1][3] = [2][4]$;
        \item $[4] = [1][4] = [2][2] = [3][3]$.
    \end{itemize}

    Lo schema appena fatto ci dice, per esempio, che se noi moltiplichiamo due elementi qualunque della classe $[4]$
    otteniamo un elemento della classe $[1]$ (per esempio: $9 \cdot 14 = 126$).
    Se invece moltiplichiamo un qualunque elemento della classe $[2]$ con un qualunque elemento delle classe $[4]$
    otteniamo un elemento della classe $[3]$ (per esempio: $7 \cdot 9 = 63$)

    Ora possiamo capire come dobbiamo comportarci se coloriamo di bianco una delle classi di congruenza:
    \begin{itemize}
        \item se coloriamo di bianco la classe $[0]$ possiamo colorare le altre classi di qualunque altro colore,
        e il vincolo è rispettato;
        \item se coloriamo di bianco la classe $[1]$ dobbiamo colorare di bianco anche la classe $[4]$ (visto che otteniamo un
        elemento di $[1]$ moltiplicando tra loro due qualunque elementi di $[4]$) e, di conseguenza, dobbiamo colorare
        di bianco anche le classi $[2]$ e $[3]$ (visto che otteniamo un elemento di $[4]$ moltiplicando due elementi di
        $[2]$ o due elementi di $[3]$); quindi se coloro di bianco $[1]$ devo colorare anche $[2]$, $[3]$ e $[4]$;
        \item se coloriamo di bianco la classe $[2]$ dobbiamo colorare di bianco anche la classe $[3]$ oppure la classe
        $[4]$; nel primo caso non ho bisogno di colorare altro di bianco;
        nel secondo caso devo colorare di bianco anche la classe $[3]$;
        \item se coloriamo di bianco la classe $[3]$ dobbiamo colorare di bianco anche la classe $[2]$ oppure la classe
        $[4]$; nel primo caso non ho bisogno di colorare altro di bianco;
        nel secondo caso devo colorare di bianco anche la classe $[2]$;
        \item se coloriamo di bianco la classe $[4]$ dobbiamo colorare di bianco anche la classe $[2]$ e $[3]$.
    \end{itemize}

    Riassumendo, abbiamo le seguenti situazioni:
    \begin{itemize}
        \item coloriamo di bianco le classi $[1]$, $[2]$, $[3]$ e $[4]$:
        possiamo colorare la classe $[0]$ di qualunque colore;
        abbiamo quindi 6 possibili scelte legate al colore che daremo alla classe $[0]$;
        \item coloriamo di bianco le classi $[2]$, $[3]$ e $[4]$ e la classe $[1]$ di un colore che non sia bianco;
        anche in questo caso possiamo colorare la classe $[0]$ di qualunque colore (compreso il bianco);
        abbiamo quindi 6 possibili scelte per il colore di $[0]$ e 5 possibile scelte (escludiamo infatti il bianco)
        per la classe $[1]$;
        \item coloriamo di bianco le classi $[2]$ e $[3]$ e le classi $[1]$ e $[4]$ di colori che non siano il bianco;
        anche in questo caso possiamo colorare la classe $[0]$ di qualunque colore (compreso il bianco);
        abbiamo quindi 6 possibili scelte per il colore di $[0]$, 5 possibili scelte per $[1]$ e 5 possibili scelte
        per $[4]$;
        \item coloriamo le classi $[1]$, $[2]$, $[3]$ e $[4]$ di una qualunque combinazione di colori che non includa
        il bianco;
        siamo ancora liberi di colorare $[0]$ con qualunque colore, compreso il bianco;
        abbiamo quindi 6 possibili scelte per il colore di $[0]$, 5 per $[1]$, 5 per $[2]$, 5 per $[3]$ e 5 per $[4]$.
    \end{itemize}

    La cosa importante è avere coscienza che le quattro casistiche viste sopra sono \emph{disgiunte}, ovvero non c'è
    nessun caso che compare più di una volta.
    Se i casi non fossero disgiunti, starei contando una possibilità più volte.

    Visto che i casi sono disgiunti, possiamo sommare tutte le possibili scelte:
    \begin{multline*}
        6 + 6 \cdot 5 + 6 \cdot 5 \cdot 5 + 6 \cdot 5 \cdot 5 \cdot 5 \cdot 5 = \\
        = 6(1 + 5 + 5^2 + 5^4) = 6(1 + 5 + 25 + 625) = 6 \cdot 656 = \\
        = 3936
    \end{multline*}

    \bigskip
    \mdfsubtitle{Risposta}
    Jacopo può colorare gli interi in 3936 modi diversi.
\end{soluzione}

\begin{soluzione}
    \mdfsubtitle{Soluzione esercizio~\ref{ex:distrettuali_2021}}
    I numeri palindromi di 6 cifre hanno la seguente struttura:

    \begin{equation*}
        \overline{abccba}
    \end{equation*}

    Dobbiamo trovare quanti di questi numeri sono divisibili per 33, ovvero sono divisibili per 3 e per 11.

    Innanzitutto osserviamo che tutti i numeri palindromi di 6 cifre sono divisibili per 11, infatti, per il
    criterio di congruenza modulo 11:

    \begin{equation*}
        [\overline{abccba}]_{11} = [-a + b - c + a - b + c]_{11} = [0]_{11}
    \end{equation*}

    Dobbiamo quindi determinare quanti numeri palindromi di 6 cifre sono divisibili per 3.
    Per il criterio di congruenza modulo 3 risulta:
    \begin{align*}
        [\overline{abccba}]_3 &= [a + b + c + c + b + a]_3 = \\
        &= [2a + 2b + 2c]_3 = \\
        &= [2(a+b+c)]_3 = \\
        &= [2]_3[a + b + c]_3 \\
        &= [2]_3[\overline{abc}]_3
    \end{align*}

    Quindi se vogliamo che $\overline{abccba}$ sia divisibile per 3, visto che 2 non lo è, è necessario che $a + b + c$
    sia divisibile per 3.

    A questo punto ti propongo due soluzioni.

    \bigskip
    \textbf{Soluzione velocissima}

    Dobbiamo determinare quanti sono i numeri di tre cifre che sono divisibili per 3.

    I numeri di 3 cifre vanno da 100 a 999 e in totale solo 900.
    Solo uno ogni tre di questi numeri è divisibile per 100, quindi quelli divisibili per 3 sono:

    \begin{equation*}
        900 : 3 = 300
    \end{equation*}

    \bigskip
    \textbf{Soluzione più lunga}

    Continuando con il criterio di congruenza modulo 3 possiamo scrivere:

    \begin{equation*}
        [a + b + c]_3 = [a]_3 + [b]_3 + [c]_3
    \end{equation*}

    Affinché tale numero sia divisibile per 3 dobbiamo scegliere dei valori di $a$, $b$ e $c$ tali che:

    \begin{equation*}
        [a]_3 + [b]_3 + [c]_3 = [0]_3
    \end{equation*}

    Dobbiamo ora andare a contare quante solo le combinazioni di $a$, $b$ e $c$ che soddisfano tale vincolo.
    Anzi: lo
    faremo in due passaggi: prima contiamo le combinazioni delle classi di congruenza $[a]_3$, $[b]_3$ e $[c]_3$ e poi
    da queste passiamo alle cifre.

    Nota che scelta la classe di congruenza per $[a]_3$ e per $[b]_3$, resta automaticamente determinata la classe di
    congruenza per $[c]_3$.
    Per esempio, se $[a]_3 = [1]_3$ e $[b]_3 = [0]$, allora $[c]_3 =  [2]_3$, infatti:

    \begin{equation*}
        [1]_3 + [0]_3 + [2]_3 = [3]_3 = [0]_3
    \end{equation*}

    Allora le possibili combinazioni (considerando le classi di congruenza) sono date dalle 3 possibili scelte di $[a]_3$
    e dalle 3 possibili scelte di $[b]_3$:

    \begin{table}[H]
        \label{tab:distrettuali_2019_1}
        \centering
        \begin{tabular}{ccc}
            \toprule
            $[a]_3$ & $[b]_3$ & $[c]_3$ \\
            \midrule
            $[0]_3$ & $[0]_3$ & $[0]_3$ \\
            $[0]_3$ & $[1]_3$ & $[1]_3$ \\
            $[0]_3$ & $[2]_3$ & $[2]_3$ \\
            $[1]_3$ & $[0]_3$ & $[2]_3$ \\
            $[1]_3$ & $[1]_3$ & $[1]_3$ \\
            $[1]_3$ & $[2]_3$ & $[0]_3$ \\
            $[2]_3$ & $[0]_3$ & $[1]_3$ \\
            $[2]_3$ & $[1]_3$ & $[0]_3$ \\
            $[2]_3$ & $[2]_3$ & $[2]_3$ \\
            \bottomrule
        \end{tabular}
    \end{table}

    Ora sostituiamo le classi di congruenza con la quantità di scelte che abbiamo per ciascuna di esse.
    Ovviamente non conteremo tutti gli infiniti numeri interi che appartengono alle classi di congruenza, ma solo le
    cifre singole.
    Dobbiamo quindi tenere conto che:
    \begin{itemize}
        \item $[0]_3 = \{0, 3, 6, 8\}$ contiene 4 elementi, ma se usata per la prima cifra $[a]_3$ non possiamo
        considerare la cifra 0 altrimenti avremo un numero di sole 5 cifre;
        \item $[1]_3 = \{1, 4, 7\}$ contiene 3 elementi;
        \item $[2]_3 = \{2, 5, 8\}$ contiene 3 elementi.
    \end{itemize}

    Inoltre, per il principio di moltiplicazione, le possibili scelte per ogni riga saranno date dal prodotto delle
    scelte delle singole cifre.
    Quindi:

    \begin{table}[H]
        \label{tab:distrettuali_2019_2}
        \centering
        \begin{tabular}{ccccccc}
            \toprule
            $[a]_3$ & $[b]_3$ & $[c]_3$ & scelte di $[a]_3$ & scelte di $[b]_3$ & scelte di $[c]_3$ & totali \\
            \midrule
            $[0]_3$ & $[0]_3$ & $[0]_3$ & 3 & 4 & 4 & 48 \\
            $[0]_3$ & $[1]_3$ & $[1]_3$ & 3 & 3 & 3 & 27 \\
            $[0]_3$ & $[2]_3$ & $[2]_3$ & 3 & 3 & 3 & 27 \\
            $[1]_3$ & $[0]_3$ & $[2]_3$ & 3 & 4 & 3 & 36 \\
            $[1]_3$ & $[1]_3$ & $[1]_3$ & 3 & 3 & 3 & 27 \\
            $[1]_3$ & $[2]_3$ & $[0]_3$ & 3 & 3 & 4 & 36 \\
            $[2]_3$ & $[0]_3$ & $[1]_3$ & 3 & 4 & 3 & 36 \\
            $[2]_3$ & $[1]_3$ & $[0]_3$ & 3 & 3 & 4 & 36 \\
            $[2]_3$ & $[2]_3$ & $[2]_3$ & 3 & 3 & 3 & 27 \\
            \midrule
            & & & & & & 300 \\
            \bottomrule
        \end{tabular}
    \end{table}

    \bigskip
    \mdfsubtitle{Risposta}
    I numeri palindromi di 6 cifre divisibili per 33 sono 300.
\end{soluzione}

\begin{soluzione}
    \mdfsubtitle{Soluzione esercizio~\ref{ex:francesco_1}}
    Il numero $\overline{abcabc}$ può essere scomposto in:
    \begin{equation*}
        \overline{abcabc} = \overline{abc} \cdot 1001 = \overline{abc} \cdot 7 \cdot 11 \cdot 13
    \end{equation*}

    Il divisore 2023 invece si scompone in:
    \begin{equation*}
        2023 = 7 \cdot 17^2
    \end{equation*}

    Se 2023 divide $\overline{abcabc}$ significa allora che:
    \begin{equation*}
        \exists k \in \Z: \overline{abc} \cdot 7 \cdot 11 \cdot 13 = k \cdot 7 \cdot 17^2
    \end{equation*}

    Possiamo semplificare per 7:
    \begin{equation*}
        \exists k \in \Z: \overline{abc} \cdot 11 \cdot 13 = k \cdot 17^2
    \end{equation*}

    Non abbiamo altri fattori in comune da semplificare, quindi è necessario che $\overline{abc}$ sia divisibile per
    $17^2$.

    Dobbiamo quindi determinare quanti numeri di tre cifre sono divisibili per $17^2 = 289$.

    289 è maggiore di 250, quindi i numeri cercati sono meno di 4:
    \begin{equation*}
        1000 : 289 < 1000 : 250 = 4
    \end{equation*}

    289 è minore di $333 \sim \frac{1000}{3}$, quindi ci sono almeno 3 multipli di 3 cifre:
    \begin{equation*}
        1000 : 289 > 1000 : \frac{1000}{3} = 3
    \end{equation*}

    \bigskip
    \mdfsubtitle{Risposta}
    I numeri nella forma $\overline{abcabc}$ divisibili per 2023 sono 3.

    Essi sono: 289289, 578578 e 867867.
\end{soluzione}

\begin{soluzione}
    \mdfsubtitle{Soluzione esercizio~\ref{ex:francesco_2}}
    Il numero $\overline{abcabc}$ può essere scomposto in:

    \begin{equation*}
        \overline{abcabc} = \overline{abc} \cdot 1001 = \overline{abc} \cdot 7 \cdot 11 \cdot 13
    \end{equation*}

    Affinché $\overline{abcabc}$ sia un quadrato perfetto è quindi necessario che $\overline{abc}$ sia divisibile per
    7, 11 e 13, ovvero deve essere divisibile per 1001.
    Ma il più piccolo numero (diverso da 0) divisibile per 1001 è 1001, che è composto di 4 cifre, non di 3.

    \bigskip
    \mdfsubtitle{Risposta}
    Non esistono numeri nella forma $\overline{abcabc}$ che sono quadrati perfetti.
\end{soluzione}


\tableofcontents

\end{document}